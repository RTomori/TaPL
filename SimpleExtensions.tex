%%% Local Variables:
%%% mode: japanese-laTeX
%%% TeX-engine: xetex
%%% End;
\documentclass[9pt]{beamer}
 \usepackage{zxjatype}
 \usepackage{xltxtra}
 \usepackage[ipa]{zxjafont}
\usepackage{amssymb, amsmath,amsfonts}
\usepackage{mathtools}
\usepackage{bussproofs}
\usepackage{mathcomp}
\usepackage{tcolorbox}
\usepackage{mathrsfs}
\tcbuselibrary{raster,skins}
\usepackage{varwidth}
\usepackage{listings}
\usepackage{bcprules}
\usetheme{metropolis}

% 色定\textmd{}義
\definecolor{mstruct}{RGB}{68, 174, 234} 
% \definecolor{malert}{RGB}{223, 153, 155}
\definecolor{malert}{RGB}{255, 76, 0}
\definecolor{mex}{RGB}{57, 149, 82}
% 見出しカラー
% block title color
% alert color
% 箇条書き
\useinnertheme{circles}
% フッダー
\setbeamertemplate{footline}[frame number]
%無を出力するコマンド
\newtcolorbox{tblock}[1]{
	enhanced, skin=enhancedlast jigsaw,
	attach boxed title to top left={xshift=-4mm,yshift=-0.5mm},
	colbacktitle=mstruct, colframe=mstruct\textmd{},
	interior style={top color=mstruct!10!white, bottom color=white},
	boxed title style={empty,arc=0pt,outer arc=0pt,boxrule=0pt},
	underlay boxed title={
		\fill[mstruct] (title.north west) -- (title.north east)
		-- +(\tcboxedtitleheight-1mm,-\tcboxedtitleheight+1mm)
		-- ([xshift=4mm,yshift=0.5mm]frame.north east) -- +(0mm,-1mm)
		-- (title.south west) -- cycle;
		\fill[mstruct!45!white!50!black] ([yshift=-0.5mm]frame.north west)
		-- +(-0.4,0) -- +(0,-0.3) -- cycle;
		\fill[mstruct!45!white!50!black] ([yshift=-0.5mm]frame.north east)
		-- +(0,-0.3) -- +(0.4,0) -- cycle; 
	},
	title=#1
}
% Definition Box
\newtcolorbox{dblock}[1]{enhanced, skin=enhancedlast jigsaw,
	attach boxed title to top left={xshift=-4mm,yshift=-0.5mm},
	colbacktitle=malert, colframe=malert,
	interior style={top color=malert!10!white, bottom color=white},
	boxed title style={empty,arc=0pt,outer arc=0pt,boxrule=0pt},
	underlay boxed title={
		\fill[malert] (title.north west) -- (title.north east)
		-- +(\tcboxedtitleheight-1mm,-\tcboxedtitleheight+1mm)
		-- ([xshift=4mm,yshift=0.5mm]frame.north east) -- +(0mm,-1mm)
		-- (title.south west) -- cycle;
		\fill[malert!45!white!50!black] ([yshift=-0.5mm]frame.north west)
		-- +(-0.4,0) -- +(0,-0.3) -- cycle;
		\fill[malert!45!white!50!black] ([yshift=-0.5mm]frame.north east)
		-- +(0,-0.3) -- +(0.4,0) -- cycle; 
	},
	title=#1
}
% subbox
\newtcolorbox{subbox}[1]{
	empty,
	coltitle=mstruct, fonttitle=\bfseries,
	borderline horizontal={0.5mm}{0pt}{mstruct},
	title=#1
	titlerule style={
		mstruct,
		arrows={Hooks[arc=270]-Hooks[arc=270]}
	}
}

\everymath{\displaystyle}
\title{Types and Programming Languages\\ Chapter 8,9}
\author{Rei Tomori}
\begin{document}
\maketitle
\begin{frame}{概要}
	今回は以下の内容を扱う: \begin{enumerate}
        \item \S 11. 単純な拡張
        \item \S 12. 正規化
        \item \S 13. 参照
    \end{enumerate}
\end{frame}
\section{\S11. 単純な拡張}
\subsection{\S11.1 基本型}
\begin{frame}{基本型}
    \begin{enumerate}
    \item プログラミング言語には基本型\footnote{基底型}.i.e.構造を持たない単純な値の集合とその上のプリミティブな演算がある.\begin{itemize}
    \item 今後は$\mathtt{Nat, Bool}$に加え,$\mathtt{String, Float}$型を用いる.
    \end{itemize}
    \item (その上の演算を捨象した)一般的な基本型を扱いたいことがある.そのためには,言語が"非解釈の"基本型の集合$\mathscr{A}$\footnote{原子型,つまり型システムにおいては内部システムを持たない型の略}を備えているとする.\begin{itemize}
    \item これを表わすためには,型の構文規則を変更してメタ変数$\mathtt{A}$($\mathscr{A}$の要素を表わす)を加える.
    \item 以降,基本型の名前として$\mathtt{A, B, C}$を用いる.
    \end{itemize}
    \item 非解釈な型を導入することで,基本型の要素上を特定することなく,その上で走る変数を束縛できる.\begin{itemize}
    \item $\mathtt{\lambda x:A. x}$は$\mathtt{x:A}$が何であれ,$\mathtt{x}$を$\mathtt{x}$自身に写す恒等関数である.
    \end{itemize}
    \end{enumerate}
\end{frame}

\begin{frame}{\S11.2. $\mathtt{Unit}$型}
    要素を 1 つしか持たない型である$\mathtt{Unit}$型を導入する.この型は次のように解釈される:\begin{itemize}
    \item 唯一の要素は項定数$\mathtt{unit}$($\mathtt{u}$,しばしば()で表わされる)で,任意の$\mathtt{Unit}$型の項は一意的に$\mathtt{unit}$に評価される.
    \end{itemize}
    $\mathtt{Unit}$型は主に副作用を持つ言語で応用される\footnote{純粋函数型言語では,たとえば各$n\in\mathbb{N}$に対して}.たとえば,可変参照を変更する関数では,返り値ではなく副作用に興味があるため,$\mathtt{Unit}$型が返り値の型とされる.類似物として C 系言語の$\mathtt{void}$型がある.
    \end{frame}
\begin{frame}{Unit 型}
$\mathtt{Unit}$型は\footnote{Haskell でいうところの 0-tuple},以下の構文,型付け規則および派生形式(i.e.糖衣構文),すなわち STLC への埋め込み方で定義される.
\begin{dblock}{Def.$\mathtt{Unit}$型の定義}
	\begin{columns}
		\begin{column}{0.30\textwidth}
			新しい構文形式
			\begin{align*}
			\mathtt{\text{(項)\ }t}\Coloneq&\ldots\\
                                        &\mathtt{Unit}
			\end{align*}
        \begin{align*}
                \mathtt{\text{(項)\ }v}\Coloneq&\ldots\\
                &\mathtt{unit}
            \end{align*}
            \begin{align*}
                \mathtt{\text{(型)\ }T}\Coloneq&\ldots\\
                &\mathtt{Unit}
            \end{align*}
		\end{column}\begin{column}{0.66\textwidth}
			新しい型付け規則\begin{prooftree}
				\AxiomC{$\Gamma\vdash \mathtt{unit:Unit}$}
			\end{prooftree}
            新しい派生形式
            \begin{align*}
                \mathtt{t_{1};t_{2}}&\stackrel{\text{def}}{=}(\lambda \mathtt{x:Unit. t_{2}})\mathtt{t_{1}}\\
                    &\text{ただし}x\notin FV(\mathtt{t_{2}})
            \end{align*}
		\end{column}
	\end{columns}\end{dblock}
\end{frame}
\begin{frame}{\S11.3.派生形式: 逐次実行とワイルドカード}
副作用のある言語における文の逐次実行を形式化する.逐次実行形式は項$\mathtt{t_{1}, t_{2}}$に対し,$\mathtt{t_{1}}$を正規形まで評価し,結果を捨てた後に$\mathtt{t_{2}}$を評価する.

$(;)$の意味を直接表わす方法と,$(;)$を内部言語のある項の略記とする 2 通りの形式化が考えられる.\begin{enumerate}\item $\mathtt{t_{1};t_{2}}$を新たな構文要素とする方法.評価規則
\begin{prooftree}
\AxiomC{$\mathtt{t_{1}\rightarrow t'_{1}}$}
\RightLabel{\tiny\rm{(E-SEQ)}}
\UnaryInfC{$\mathtt{t_{1};t_{2}\rightarrow t'_{1};t_{2}}$}
\end{prooftree}
\begin{prooftree}
        \AxiomC{}
        \noLine
        \UnaryInfC{$\mathtt{v_{1};t_{2}\rightarrow t_{2}}$\ \tiny\rm{(E-SEQNEXT)}}
\end{prooftree}
および型付け規則
\begin{prooftree}
\AxiomC{$\Gamma\vdash\mathtt{t_{1}:Unit}$}
\AxiomC{$\Gamma\vdash\mathtt{t_{2}:T_{2}}$}
\RightLabel{\tiny\rm{(T-SEQ)}}
\BinaryInfC{$\Gamma\vdash\mathtt{t_{1};t_{2}:T_{2}}$}
\end{prooftree}を付け加えることで$(;)$の振る舞いを特徴付ける.
\item 内部言語の項の略記とする方法.$\mathtt{t_{1};t_{2}\stackrel{def}{=}(\lambda x:Unit. t_{2})t_{1}}$,ただし$x\mathtt{x}\notin FV(\mathtt{t_{2}})$とする.
\end{enumerate}
\end{frame}
\begin{frame}{派生形式}
はじめの形式化で定めた評価および型付け規則は,$\mathtt{Unit}$のみを型として持つ STLC の評価関係および型付け規則より従う.このことを確認しよう.
\begin{alertblock}{Thm.11.3.1[逐次実行は派生形式である]}
    $\mathtt{Unit}$型,逐次実行およびそれらの評価・型付け規則を持つ STLC を$\lambda^{E}$と書く.\footnote{外部言語の略.}また,$\mathtt{Unit}$型のみを持つ STLC を$\lambda^{I}$と書き,$e\in\lambda^{I}\to\lambda^{E}$を,$\lambda^{I}$の各項を対応する$\lambda^{E}$の項に写す詳細化関数\footnote{elaboration function}とする.つまり,$e$は$\mathtt{t_{1};t_{2}}$の各出現を,$\mathtt{(\lambda x:Unit. t_{2})t_{1}}(\mathtt{x}\notin FV(\mathtt{t_{2}}))$に置き換える.すると,$\lambda^{E}$の各項$\mathtt{t}$に対して,\begin{enumerate}
    \item $\mathtt{t\rightarrow_{E} t'}\Rightarrow e(\mathtt{t})\rightarrow_{I} e(\mathtt{t'})$.逆に$e(\mathtt{t})\rightarrow_{I}\mathtt{u}\Rightarrow\exists \mathtt{t'}.\mathtt{t'}$は$\lambda^{E}$の項$\land \mathtt{u} = e(\mathtt{t'})\land \mathtt{t\rightarrow_{E}t'}$.
    \item $\Gamma\vdash^{E}\mathtt{t:T}\Leftrightarrow \Gamma\vdash^{I}e\mathtt{(t):T}$
    \end{enumerate}
\end{alertblock}
\end{frame}
\begin{frame}{Thm.11.3.1.の証明(一部)}
    \begin{proof}
       $\mathtt{t}$の構造帰納法で示す.(2)の主張は白板で示すことにする.
        \begin{enumerate}
        \item (1)の証明.\begin{enumerate}
        \item $(\Rightarrow)$ 新たな構文要素,つまり$\mathtt{t = t_{1};t_{2}}$または$\mathtt{t = v_{1};t_{2}}$の場合を考え,$\mathtt{t\rightarrow_{E} t'}$なる$\mathtt{t'}$の存在を仮定.\begin{itemize}
        \item $\mathtt{t = v_{1};t_{2}}$.仮定より$\mathtt{t' = t_{2}}$.いま$\mathtt{x}\notin FV(\mathtt{t_{2}})$を任意に取ると,定義より$e(\mathtt{t}) = \mathtt{(\lambda x: Unit. }e\mathtt{(t_{2}))\ v_{1}}$.$\text{\rm(E-APPABS)}$から$\mathtt{t}\rightarrow_{I}\left[\mathtt{x\mapsto v_{1}}\right]e\mathtt{(t_{2})} = \mathtt{t'}$がいえ,$\mathtt{x\notin}FV(e(\mathtt{t_{2}}))$からこれは$\mathtt{v_{1}}$に等しい.
        \item $\mathtt{t = t_{1};t_{2}}$.仮定より$\mathtt{t' = t'_{1};t_{2}}\land \mathtt{t_{1}\rightarrow_{E}t'_{1}}$なる$\mathtt{t', t'_{1}}$が存在.IH より$e(\mathtt{t_{1}}) \rightarrow_{I} e(\mathtt{t'_{1}})$.$\text{\rm{(E-APP1)}}$から$e(\mathtt{t}) = e(\mathtt{t_{1}})e(\mathtt{t_{2}})\rightarrow_{I} e(\mathtt{t'_{1}})e(\mathtt{t_{2}} = e(\mathtt{t'})$.
        \end{itemize}
        \item $(\Leftarrow)$.$e(\mathtt{t})\rightarrow_{I}\mathtt{u}$を仮定する.(1)と同様に場合を分ける.\begin{itemize}
        \item $\mathtt{t = v_{1};t_{2}}$のとき.$e(\mathtt{t}) = \mathtt{(\lambda x:Unit. }e\mathtt{(t_{2}))v_{1}}$($\mathtt{x}$は fresh な変数),$\mathtt{u} = e(\mathtt{t_{2}})$.このとき.$\mathtt{t' = t_{2}}$とすれば従う.
        \item $\mathtt{t = t_{1};t_{2}}$のとき.$\mathtt{t} = (\mathtt{\lambda x:Unit.}e (\mathtt{t_{2}}))\ e(\mathtt{t_{1}})$であり,仮定より$e(\mathtt{t_{1}})\rightarrow \mathtt{u'_{1}}$なる$\mathtt{u'_{1}}$を取ると$\mathtt{u} = (\mathtt{\lambda x:Unit.}e(\mathtt{t_{1}}))\ \mathtt{u'_{1}}$.帰納法の仮定より,$\mathtt{u'_{1}} = e(\mathtt{t'_{1}})\land \mathtt{t_{1}}\rightarrow_{E} \mathtt{t'_{1}}$なる$\lambda^{E}$の項$\mathtt{t'_{1}}$が取れる.ゆえに$\mathtt{u} = (\mathtt{\lambda x: Unit.}e(\mathtt{t_{2}}))e(\mathtt{t'_{1}}) = e(\mathtt{t'_{1};t_{2}})$となり従う.
        \end{itemize}
        \end{enumerate}
        \end{enumerate}
    \end{proof}
\end{frame}
\begin{frame}{派生形式}
\begin{enumerate}\item 派生形式として導入する利点には,表層構文を拡張しつつ型安全性を保証しなければならない内部言語を単純に保てることにある.
    \item 他の派生形式として,抽象の本体で使わない引数を束縛するようなワイルドカード\footnote{hole などということもある}の慣習がある.ワイルドカード束縛子は$(\_)$で表わされる.
    \begin{itemize}
    \item $\mathtt{\lambda \_: S. t\stackrel{def}{=}\lambda x: S. t}$,ただし$\mathtt{x}$は$\mathtt{t}$に現れない.
    \end{itemize}
\end{enumerate}
\end{frame}
\begin{frame}{\S11.4.型指定}
\begin{enumerate}
\item 所与の項に特定の型を明示的に指定する機能を\underline{型指定}といい,型$\mathtt{T}$を指定した項$\mathtt{t}$は$\mathtt{t\ as\ T}$と書かれる.用途としては次のようなものがある:
\begin{itemize}
\item ドキュメンテーション.特に型シグネチャが複雑になる場合などは有用.
\item 複雑な型の表示の制御.複雑な型式の略記\footnote{型シノニムなどと呼ばれる}を導入し,型検査器に必要に応じて略記を折り畳み/展開させる.
\item 抽象化.項$\mathtt{t}$が複数の異なる型を持ちうる体系(たとえば部分型付けのある)において,$\mathtt{t}$がより少数の型を持つかのよう型検査器に指示し,型の一部を隠蔽させる.
\end{itemize}
\end{enumerate}
\end{frame}
\begin{frame}{型指定}
型指定の型付けおよび評価規則は以下の通り:
\begin{alertblock}{Def.型指定}
    \begin{columns}
    \begin{column}{0.30\columnwidth}
        \begin{align*}
        \mathtt{t}\Coloneq&\ldots\\
        &\mathtt{t\ as\ T}
        \end{align*}
        評価規則
        \begin{prooftree}
            \AxiomC{$\mathtt{v_{1}\ as\ T\rightarrow v_{1}}$\text{\tiny\rm{(E-ASCRIBE)}}}
        \end{prooftree}
        \begin{prooftree}
        \AxiomC{$\mathtt{t_{1}\rightarrow t'_{1}}$}
        \RightLabel{\tiny\rm{(E-ASCRIBE1)}}
        \UnaryInfC{$\mathtt{t_{1}\ as\ T\rightarrow t'_{1}\ as\ T}$}
        \end{prooftree}
    \end{column}
    \begin{column}{0.40\columnwidth}
        型付け規則
        \begin{prooftree}
            \AxiomC{$\Gamma\vdash\mathtt{t_{1}:T}$}
            \RightLabel{\tiny\rm{(T-ASCRIBE)}}
            \UnaryInfC{$\Gamma\vdash\mathtt{t_{1}\ as\ T:T}$}
        \end{prooftree}
    \end{column}
    \end{columns}
\end{alertblock}
\end{frame}
\begin{frame}{型指定}
\begin{alertblock}{演習 11.4.1.(推奨)}
    \begin{enumerate}
    \item 型指定を派生形式として形式化し,p.10 の型付け規則と評価規則が派生形式と対応することを示せ.
    \item $\text{\rm{E-ASCRIBE, E-ASCRIBE1}}$の代りに,以下の規則が与えられたとする:\begin{prooftree}
        \AxiomC{$\mathtt{t_{1}\ as\ T\rightarrow t_{1}}$\text{\tiny\rm{(E-ASCRIBEEAGER)}}}
    \end{prooftree}
    この場合に型指定を派生形式として扱うことは出来るか.
    \end{enumerate}
\end{alertblock}
\begin{proof}
    \begin{enumerate}
    \item $\mathtt{t\ as\ T}$を$\mathtt{(\lambda x : T. x)\ t}$の略記と定める.評価規則と型付け規則を示そう.\begin{itemize}
    \item 評価規則.$\mathtt{v_{1}\ as\ T}$とすると$\mathtt{v_{1}\ as\ T} = \mathtt{(\lambda x : T. x)\ v_{1}} \rightarrow \left[x\mapsto v_{1}\right] \mathtt{x} = \mathtt{v_{1}}$より$\text{\tiny\rm{E-ASCRIBE}}$が従う.$\mathtt{t_{1}\rightarrow t'_{1}}$とすると$\mathtt{t_{1}\ as\ T} = \mathtt{(\lambda x:T. x)\ t_{1}}\rightarrow\mathtt{(\lambda x: T. x)\ t'_{1}} = \mathtt{t'_{1}\ as\ T}$より,$\text{\tiny\rm{E-ASCRIBE1}}$も従う.
    \item 型付け規則.$\Gamma\vdash\mathtt{t_{1}:T}$とする.
    \end{itemize}
    \end{enumerate}
\end{proof}
\end{frame}
\begin{frame}{演習 11.4.1.}
\begin{proof}
\begin{enumerate}
\item 以下の導出木より成り立つ:\begin{prooftree}
\AxiomC{$\Gamma, \mathtt{x:T}\vdash \mathtt{x:T}$}
\RightLabel{\tiny\rm{(Proj)}}
\UnaryInfC{$\Gamma\vdash\mathtt{\lambda x: T. x:T\rightarrow T}$}
\RightLabel{\tiny\rm{(T-ABS)}}
\AxiomC{$\Gamma\vdash\mathtt{t_{1}:T}$}
\BinaryInfC{$\Gamma\vdash \mathtt{(\lambda x : T. x)\ t_{1}:T}$}
\UnaryInfC{$\Gamma\vdash \mathtt{t_{1}\ as\ T: T}$}
\end{prooftree}
\item $\text{\tiny\rm{E-ASCRIBEEAGER}}$を実現するためにはサンクを噛ませることで$\mathtt{t_{1}}$の評価を遅延させなければならない.次のようになる.ただし$\mathtt{y}$はフレッシュとする:\begin{equation*}
\mathtt{t_{1}\ as\ T \stackrel{def}{=} (\lambda x: Unit\rightarrow T. x\ unit)(\lambda y:Unit.  t_{1})}
\end{equation*}
この定義では,$\mathtt{t\ as\ T}$の型指定を消去するために$\mathtt{t\ as\ T\rightarrow (\lambda y : Unit. t_{1})\ unit\rightarrow t_{1}}$と 2 ステップの簡約を経なければならない.したがって,Thm11.3.1 を考えると,(2)は成立するものの,(1)の結論は内部言語で多ステップの簡約を経るように弱めなければならない.
\end{enumerate}
\end{proof}
\end{frame}
\begin{frame}{\S11.5.let 束縛}
\begin{enumerate}
\item ML 系言語などでは,部分式に名前を付ける手段として$\mathtt{let}$束縛子を提供している.\begin{itemize}
\item $\mathtt{let\ x = t_{1}\ in\ t_{2}}$は,項$\mathtt{t_{1}}$を値に評価して$\mathtt{x}$に束縛し,$\mathtt{t_{2}}$を評価することを意味していた.
\end{itemize}\end{enumerate}
今回は ML 同様に値呼び戦略を用いて$\mathtt{let}$束縛子を定義する:
\begin{alertblock}{Def.11.4.$\mathtt{let}$束縛}
    \begin{columns}
    \begin{column}{0.3\columnwidth}
        構文要素
        \begin{align*}
        \mathtt{t}\Coloneq&\ldots\\
            &\mathtt{let\ x = t_{1}\ in\ t_{2}}
        \end{align*}
    \end{column}
    \begin{column}{0.65\columnwidth}
        評価関係
        \begin{prooftree}
            \AxiomC{$\mathtt{let\ x = v_{1}\ in\ t_{2}\rightarrow \left[x\mapsto v_{1}\right]t_{2}}$\text{\tiny\rm{(E-LETV)}}}
            \end{prooftree}
            \begin{prooftree}
                \AxiomC{$\mathtt{t_{1}\rightarrow t'_{1}}$}
                \RightLabel{\tiny\rm{(E-LET)}}
                \UnaryInfC{$\mathtt{let\ x = t_{1}\ in\ t_{2}\rightarrow let\ x=t'_{1}\ in\ t_{2}}$}
                \end{prooftree}
        型付け規則
        \begin{prooftree}
        \AxiomC{$\Gamma\vdash \mathtt{t_{1}: T_{1}}$}
        \AxiomC{$\Gamma,\mathtt{x:T_{1}}\vdash\mathtt{t_{2}:T_{2}}$}
        \RightLabel{\tiny\rm{(T-LET)}}
        \BinaryInfC{$\Gamma\vdash\mathtt{let\ x = t_{1}\ in\ t_{2}:T_{2}}$}
        \end{prooftree}
    \end{column}
    \end{columns}
\end{alertblock}
\end{frame}
\begin{frame}{演習 11.5.1}
\begin{alertblock}{演習 11.5.1.(推奨)}
    型検査器$\mathtt{letexercise}$の$\mathtt{eval}$関数および$\mathtt{typeof}$関数を完成させよ.
\end{alertblock}
\begin{proof}[解答]
    $\mathtt{eval1}$関数のパターンマッチの$\mathtt{TmLet}$の節に以下を足せばよい:
    \begin{align*}
    \mathtt{  |TmLet(fi, x, t1, t2)} &\rightarrow \mathtt{if\ isval\ ctx \ t1\ then\ termSubstTop\ t1\ t2}\\
        &\mathtt{else\ let\ t1' = eval1\ ctx\ t1\ in\ TmLet(fi, x, t1', t2)}
    \end{align*}
    型検査器$\mathtt{typeof}$関数に対しては以下を足せばよい:
\end{proof}
\end{frame}
\begin{frame}{let 束縛}
$\mathtt{let}$を派生形式として定義することを考える.\begin{itemize}
\item 素朴には,$\mathtt{let\ x= t_{1}\ in \ t_{2}}\stackrel{def}{=}\mathtt{(\lambda x:T_{1}. t_{1})\ t_{2}}$とすることが考えられるが,左辺には右辺の型注釈が出現しない(実際には,構文解析器は型検査器から$\mathtt{x}$の型を得る).そのため,単純に項に対する脱糖衣変換とはみなせず,代わりに型付け導出に対する変換と見做す必要がある.具体的には,派生形式は$\mathtt{let}$に関する導出\begin{prooftree}
\AxiomC{$\vdots$}
\UnaryInfC{$\Gamma\vdash\mathtt{t_{1}:T_{1}}$}
\AxiomC{$\vdots$}
\UnaryInfC{$\Gamma,\mathtt{x:T_{1}}\vdash\mathtt{t_{2}:T_{2}}$}
\BinaryInfC{$\Gamma\vdash\mathtt{let\ x= t_{1}\ in\ t_{2}:T_{2}}$}
\end{prooftree}
を以下の導出に写すように定める:\begin{prooftree}
\AxiomC{$\vdots$}
\UnaryInfC{$\Gamma, \mathtt{x:T_{1}}\vdash\mathtt{t_{2}:T_{2}}$}
\RightLabel{\tiny\rm{(T-ABS)}}
\UnaryInfC{$\Gamma\vdash \mathtt{\lambda x: T_{1}.t_{2}:T_{2}}$}
\AxiomC{$\vdots$}
\UnaryInfC{$\Gamma\vdash\mathtt{t_{1}:T_{1}}$}
\RightLabel{\tiny\rm{(T-APP)}}
\BinaryInfC{$\Gamma\vdash\mathtt{(\lambda x:T_{1}.t_{2})t_{1}:T_{2}}$}
\end{prooftree}
\end{itemize}
つまり,評価に関する振る舞いとは違い,型付けに関する振る舞いは(型が implicit な場合を扱えるように)内部言語に組込まれていなければならない.このことは,\S22 でより詳細に扱う.
\end{frame}
\begin{frame}{\S11.6.2 つ組}
\begin{enumerate}
\item 2 つ組,ひいては任意個の値の組(およびその一般化であるレコード型)について考える.\footnote{以後,$n$個組を単にタプルと呼ぶことがある.}\begin{itemize}
\item 以下の構文要素を追加する:2 つ組化$\left\{\mathtt{t_{1},t_{2}}\right\}$,第一および第二成分への射影$\mathtt{t.1, t.2}$,および型$\mathtt{T_{1}}$と$\mathtt{T_{2}}$の直積\footnote{デカルト積}$\mathtt{T_{1}\times T_{2}}$.組を表現するために波括弧を用いているのはレコード型の具体例であることを強調するため.
\item 組は第一成分を値に評価した後第二成分を,そして完全に評価された 2 つ組に対して射影の適用を許すことにする.
\end{itemize}
\end{enumerate}
型付け規則は次頁に示す通り:
\end{frame}
\begin{frame}{\S11.6.2 つ組}
\begin{columns}
\begin{column}{0.2\columnwidth}
    構文:$\mathtt{t\Coloneq\ldots|\left\{t, t\right\}|t.1|t.2}$

    値:$\mathtt{v\Coloneq\ldots\left\{v, v\right\}}$

    型:$\mathtt{T\Coloneq \ldots|T_{1}\times T_{2}}$
評価・型付け
\begin{prooftree}
    \AxiomC{$\mathtt{\left\{v_{1},v_{2}\right\}}.1\rightarrow\mathtt{v_{1}}$}
    \end{prooftree}
    \begin{prooftree}
        \AxiomC{$\mathtt{\left\{v_{1},v_{2}\right\}}.2\rightarrow\mathtt{v_{2}}$}
        \end{prooftree}
\end{column}
\begin{column}{0.75\columnwidth}
    \begin{prooftree}
    \AxiomC{$\mathtt{t_{1}\rightarrow t'_{1}}$}
    \RightLabel{\tiny\rm{(E-PROJ1)}}
    \UnaryInfC{$\mathtt{t_{1}.1\rightarrow t'_{1}.1}$}
    \end{prooftree}
    \begin{prooftree}
        \AxiomC{$\mathtt{t_{1}\rightarrow t'_{1}}$}
        \RightLabel{\tiny\rm{(E-PROJ2)}}
        \UnaryInfC{$\mathtt{t_{1}.2\rightarrow t'_{1}.2}$}
        \end{prooftree}
    \begin{prooftree}
    \AxiomC{$\mathtt{t_{1}\rightarrow t'_{1}}$}
    \RightLabel{\tiny\rm{(E-PAIR1)}}
    \UnaryInfC{$\mathtt{\left\{t_{1}, t_{2}\right\}\rightarrow \left\{t'_{1},t_{2}\right\}}$}
    \end{prooftree}
    \begin{prooftree}
        \AxiomC{$\mathtt{t_{2}\rightarrow t'_{2}}$}
        \RightLabel{\tiny\rm{(E-PAIR2)}}
        \UnaryInfC{$\mathtt{\left\{v_{1}, t_{2}\right\}\rightarrow \left\{v_{1},t'_{2}\right\}}$}
        \end{prooftree}
        \begin{prooftree}
        \AxiomC{$\Gamma\vdash\mathtt{t_{1}:T_{1}}$}
        \AxiomC{$\Gamma\vdash\mathtt{t_{2}:T_{2}}$}
        \RightLabel{\tiny\rm{(T-PAIR)}}
        \BinaryInfC{$\Gamma\vdash \mathtt{\left\{t_{1}, t_{2}\right\}:T_{1}\times T_{2}}$}
        \end{prooftree}
        \begin{prooftree}
        \AxiomC{$\Gamma\vdash\mathtt{t_{1}:T_{1}\times T_{2}}$}
        \RightLabel{\tiny\rm{T-PROJ1}}
        \UnaryInfC{$\Gamma\vdash\mathtt{t_{1}.1:T_{1}}$}
        \end{prooftree}
        \begin{prooftree}
            \AxiomC{$\Gamma\vdash\mathtt{t_{1}:T_{1}\times T_{2}}$}
            \RightLabel{\tiny\rm{T-PROJ2}}
            \UnaryInfC{$\Gamma\vdash\mathtt{t_{1}.2:T_{2}}$}
            \end{prooftree}
\end{column}
\end{columns}
\end{frame}
\begin{frame}{\S11.7.組}
\S11.6 を$n$項の直積に拡張する.$n$項の直積を,$\left\{\mathtt{t_{i}^{i\in 1\ldots n}}\right\}:\left\{\mathtt{T_{i}^{i\in 1\ldots n}}\right\}$と表わす.ただし$n = 0$の場合は空の組$\left\{\right\}$を表わし,射影は定義されない.構文,評価および型付けは以下の通り:\begin{alertblock}{Def.11.6.組}
\begin{columns}
\begin{column}{0.18\columnwidth}
項:$\mathtt{t\Coloneq\ldots|\left\{t_{i}^{i\in 1\ldots n}\right\}|t.i}$

値:$\mathtt{v\Coloneq\ldots|\left\{v_{i}^{i\in 1\ldots n}\right\}}$

型:$\mathtt{T\Coloneq\ldots|\left\{T_{i}^{i \in 1\ldots n}\right\}}$

評価規則:
\begin{prooftree}
\AxiomC{$\mathtt{\left\{v_{i}^{i\in 1\ldots n}\right\}.i\rightarrow v_{i}}$\tiny\rm{(E-PROJTUPLE)}}
\end{prooftree}
\end{column}
\begin{column}{0.7\columnwidth}
    \begin{prooftree}
        \AxiomC{$\mathtt{t_{1}\rightarrow t'_{1}}$}
        \RightLabel{\tiny\rm{(E-PROJ)}}
        \UnaryInfC{$\mathtt{t_{1}.i\rightarrow t'_{1}.i}$}
        \end{prooftree}
\begin{prooftree}
\AxiomC{$\mathtt{t_{j}\rightarrow t'_{j}}$}
\UnaryInfC{$\left\{t_{i}^{i\in 1\ldots j - 1}, t_{j}, t_{k}^{k\in j + 1\ldots n}\right\}\rightarrow\left\{t_{i}^{i\in 1\ldots j - 1}, t'_{j}, t_{k}^{k\in j + 1\ldots n}\right\}$}
\end{prooftree}
\begin{prooftree}
\AxiomC{$\forall i\in\left\{1,\ldots, n\right\}, \Gamma\vdash\mathtt{t_{i}:T_{i}}$}
\UnaryInfC{$\Gamma\vdash\mathtt{\left\{t_{i}^{i\in 1\ldots n}\right\}:\left\{T_{i}^{i\in 1\ldots n}\right\}}$}
\end{prooftree}
\begin{prooftree}
\AxiomC{$\Gamma\vdash\mathtt{t_{1}:\left\{T_{i}^{i\in 1\ldots n}\right\}}$}
\UnaryInfC{$\Gamma\vdash \mathtt{t_{1}.j : T_{j}}$}
\end{prooftree}
\end{column}
\end{columns}
\end{alertblock}
\end{frame}
\begin{frame}{\S11.8.レコード}
\S11.7 を(ラベル付き)レコード型に拡張する.以下のように定めればよい:
\begin{alertblock}{Def.11.7.}
    \begin{columns}
    \begin{column}{0.3\columnwidth}
        項:$\mathtt{t\Coloneq\ldots|\left\{l_{i} = t_{i}^{i = 1\ldots n}\right\}|t.l}$
        
        値:$\mathtt{v\Coloneq\ldots|\left\{l_{i} = v_{i}^{i\in 1\ldots n}\right\}}$

        型:$\mathtt{T\Coloneq\ldots|\left\{l_{i}:T_{i}^{i\in 1\ldots n}\right\}}$

        評価規則:
        \begin{prooftree}
        \AxiomC{$\mathtt{\left\{l_{i} = v_{i}^{i\in 1\ldots n}\right\}.l_{j}\rightarrow v_{j}}$}
        \end{prooftree}
    \end{column}
    \begin{column}{0.7\columnwidth}
        \begin{prooftree}
            \AxiomC{$\mathtt{t_{1}\rightarrow t'_{1}}$}
            \RightLabel{\tiny\rm{(E-PROJ)}}
            \UnaryInfC{$\mathtt{t_{1}.l\rightarrow t'_{1}.l}$}
            \end{prooftree}
    \begin{prooftree}
    \AxiomC{$\mathtt{t_{j}\rightarrow t'_{j}}$}
    \UnaryInfC{$\mathtt{\left\{l_{i} = t_{i}^{i <  j}, l_{j} = t_{j}, l_{k} = t_{k}^{j + 1\leq k \leq n}\right\}\rightarrow\left\{l_{i} = t_{i}^{i\in 1\ldots j - 1}, l_{j} = t'_{j}, l_{k} = t_{k}^{k\in j + 1\ldots n}\right\}}$}
    \end{prooftree}
    \begin{prooftree}
    \AxiomC{$\forall i\in\left\{1,\ldots, n\right\}, \Gamma\vdash\mathtt{t_{i}:T_{i}}$}
    \UnaryInfC{$\Gamma\vdash\mathtt{\left\{t_{i}^{i\in 1\ldots n}\right\}:\left\{T_{i}^{i\in 1\ldots n}\right\}}$}
    \end{prooftree}
    \begin{prooftree}
    \AxiomC{$\Gamma\vdash\mathtt{t_{1}:\left\{T_{i}^{i\in 1\ldots n}\right\}}$}
    \UnaryInfC{$\Gamma\vdash \mathtt{t_{1}.j : T_{j}}$}
    \end{prooftree}
    \end{column}
    \end{columns}
\end{alertblock}
\end{frame}
\begin{frame}{\S11.10.和}
\begin{enumerate}
\item 異種の値が混在したデータ構造を表わすことを考える.\begin{itemize}
\item e.g.二分木,cons-list ,$\lambda$項の抽象構文木.
\end{itemize}
\item これらのデータ構造はバリアント型を用いて表わされる.\begin{itemize}\item 以降では,まず一般的なバリアント型について議論し,ついでより制限された項である直和型,オプション型や列挙型について議論する.\end{itemize}
\end{enumerate}
\end{frame}
\begin{frame}{\S11.10.バリアント}
ラベル付きのバリアントを導入する.たとえば,以下の型を用いて 2 種類のアドレス帳のレコードを表わしているとする:
$\mathtt{PhysicalAddr = \left\{firstlast: String, addr:String\right\}}$

$\mathtt{VirtualAddr = \left\{name:String, email:String\right\}}$
\end{frame}
\begin{frame}{ヴァリアント型の構文形式}
ヴァリアントの構文形式として,次を導入する:\begin{enumerate}
\item 項:$\mathtt{t\Coloneq\ldots|\left<l = t\right>\ as\ T|case\ t\ of\ \left<l_{i}=x_{i}\right>\Rightarrow t_{i}^{i\in1\ldots n}}$
\item 値:$\mathtt{v\Coloneq\ldots|\left<l = v\right>\ as\ T}$
\item 型:$\mathtt{T\Coloneq\ldots|\left<l_{i}:T_{i}^{i\in1\ldots n}\right>}$
\end{enumerate}
\end{frame}
\begin{frame}{\S11.10.バリアント型の評価および型付け規則}
バリアント型の評価規則および型付け規則を次に示す.
\infax[E-CaseVariant]{case(\left<l_{j} = v_{j}\right>\ as\ T)\ of\ <l_{i} =x_{i}> \Rightarrow t_{i}^{i\in1\ldots n}\\\longrightarrow \left[x_{j}\mapsto v_{j}\right]t_{j}}
\infrule[E-Case]{t_{0}\rightarrow t'_{0}}{case\ t_{0}\ of \left<l_{i}=x_{i}\right>\Rightarrow t_{i}^{1\in1\ldots n}\\\longrightarrow case\ t'_{0}\ of\ \left<l_{i}=x_{i}\right>\Rightarrow t_{i}^{i\in1\ldots n}}
\infrule[T-Variant]{\Gamma\vdash t_{j}:T_{j}}{\Gamma\vdash\left<l_{j}=t_{j}\right>\ as \left<l_{i}:T_{i}^{i\in1\ldots n}\right>:\left<l_{i}:T_{i}^{i\in1\ldots n}\right>}
\infrule[T-Case]{\Gamma\vdash t_{0}:\left<l_{i}:T_{i}^{i\in1\ldots n}\right> \andalso \forall i. \Gamma, x_{i}:T_{i}\vdash t_{i}:T}{\Gamma\vdash case\ t_{0}\ of\ \left<l_{i}=x_{i}\right>\Rightarrow t_{i}^{i\in1\ldots n}:T}
\end{frame}
\end{document}
