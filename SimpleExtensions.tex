%%% Local Variables:
%%% mode: japanese-laTeX
%%% TeX-engine: xetex
%%% End;
\documentclass[9pt]{beamer}
 \usepackage{zxjatype}
 \usepackage{xltxtra}
 \usepackage[ipa]{zxjafont}
\usepackage{amssymb, amsmath,amsfonts}
\usepackage{mathtools}
\usepackage{bussproofs}
\usepackage{mathcomp}
\usepackage{tcolorbox}
\usepackage{mathrsfs}
\tcbuselibrary{raster,skins}
\usepackage{varwidth}
\usepackage{listings}
\usepackage{manfnt}
\usepackage{array}
\usepackage{bcprules}
\usetheme{metropolis}

% 色定\textmd{}義
\definecolor{mstruct}{RGB}{68, 174, 234} 
% \definecolor{malert}{RGB}{223, 153, 155}
\definecolor{malert}{RGB}{255, 76, 0}
\definecolor{mex}{RGB}{57, 149, 82}
% 見出しカラー
% block title color
% alert color
% 箇条書き
\useinnertheme{circles}
% フッダー
\setbeamertemplate{footline}[frame number]
%無を出力するコマンド
\newtcolorbox{tblock}[1]{
	enhanced, skin=enhancedlast jigsaw,
	attach boxed title to top left={xshift=-4mm,yshift=-0.5mm},
	colbacktitle=mstruct, colframe=mstruct\textmd{},
	interior style={top color=mstruct!10!white, bottom color=white},
	boxed title style={empty,arc=0pt,outer arc=0pt,boxrule=0pt},
	underlay boxed title={
		\fill[mstruct] (title.north west) -- (title.north east)
		-- +(\tcboxedtitleheight-1mm,-\tcboxedtitleheight+1mm)
		-- ([xshift=4mm,yshift=0.5mm]frame.north east) -- +(0mm,-1mm)
		-- (title.south west) -- cycle;
		\fill[mstruct!45!white!50!black] ([yshift=-0.5mm]frame.north west)
		-- +(-0.4,0) -- +(0,-0.3) -- cycle;
		\fill[mstruct!45!white!50!black] ([yshift=-0.5mm]frame.north east)
		-- +(0,-0.3) -- +(0.4,0) -- cycle; 
	},
	title=#1
}
% Definition Box
\newtcolorbox{dblock}[1]{enhanced, skin=enhancedlast jigsaw,
	attach boxed title to top left={xshift=-4mm,yshift=-0.5mm},
	colbacktitle=malert, colframe=malert,
	interior style={top color=malert!10!white, bottom color=white},
	boxed title style={empty,arc=0pt,outer arc=0pt,boxrule=0pt},
	underlay boxed title={
		\fill[malert] (title.north west) -- (title.north east)
		-- +(\tcboxedtitleheight-1mm,-\tcboxedtitleheight+1mm)
		-- ([xshift=4mm,yshift=0.5mm]frame.north east) -- +(0mm,-1mm)
		-- (title.south west) -- cycle;
		\fill[malert!45!white!50!black] ([yshift=-0.5mm]frame.north west)
		-- +(-0.4,0) -- +(0,-0.3) -- cycle;
		\fill[malert!45!white!50!black] ([yshift=-0.5mm]frame.north east)
		-- +(0,-0.3) -- +(0.4,0) -- cycle; 
	},
	title=#1
}
% subbox
\newtcolorbox{subbox}[1]{
	empty,
	coltitle=mstruct, fonttitle=\bfseries,
	borderline horizontal={0.5mm}{0pt}{mstruct},
	title=#1
	titlerule style={
		mstruct,
		arrows={Hooks[arc=270]-Hooks[arc=270]}
	}
}

\newcommand{\dnote}[1]{%
    \noindent % I guess this is intended...
    \begin{tabular}{@{}m{0.13\textwidth}@{}m{0.87\textwidth}@{}}%
        \huge\textdbend &#1%
    \end{tabular}%
    \par % ... and this too.
}
\everymath{\displaystyle}
\title{Types and Programming Languages\\ Chapter 11-13}
\author{Rei Tomori}
\begin{document}
\maketitle
\begin{frame}{概要}
	今回は以下の内容を扱う: \begin{enumerate}
        \item \S 11. 単純な拡張(\S11.4-):p.9 から
        \item \S 12. 正規化
        \item \S 13. 参照
    \end{enumerate}
\end{frame}
\section{\S11. 単純な拡張}
\subsection{\S11.1 基本型}
\begin{frame}{基本型}
    \begin{enumerate}
    \item プログラミング言語には基本型\footnote{基底型}.i.e.構造を持たない単純な値の集合とその上のプリミティブな演算がある.\begin{itemize}
    \item 今後は$\mathtt{Nat, Bool}$に加え,$\mathtt{String, Float}$型を用いる.
    \end{itemize}
    \item (その上の演算を捨象した)一般的な基本型を扱いたいことがある.そのためには,言語が"非解釈の"基本型の集合$\mathscr{A}$\footnote{原子型,つまり型システムにおいては内部システムを持たない型の略}を備えているとする.\begin{itemize}
    \item これを表わすためには,型の構文規則を変更してメタ変数$\mathtt{A}$($\mathscr{A}$の要素を表わす)を加える.
    \item 以降,基本型の名前として$\mathtt{A, B, C}$を用いる.
    \end{itemize}
    \item 非解釈な型を導入することで,基本型の要素上を特定することなく,その上で走る変数を束縛できる.\begin{itemize}
    \item $\mathtt{\lambda x:A. x}$は$\mathtt{x:A}$が何であれ,$\mathtt{x}$を$\mathtt{x}$自身に写す恒等関数である.
    \end{itemize}
    \end{enumerate}
\end{frame}

\begin{frame}{\S11.2. $\mathtt{Unit}$型}
    要素を 1 つしか持たない型である$\mathtt{Unit}$型を導入する.この型は次のように解釈される:\begin{itemize}
    \item 唯一の要素は項定数$\mathtt{unit}$($\mathtt{u}$,しばしば()で表わされる)で,任意の$\mathtt{Unit}$型の項は一意的に$\mathtt{unit}$に評価される.
    \end{itemize}
    $\mathtt{Unit}$型は主に副作用を持つ言語で応用される\footnote{純粋函数型言語では,たとえば各$n\in\mathbb{N}$に対して}.たとえば,可変参照を変更する関数では,返り値ではなく副作用に興味があるため,$\mathtt{Unit}$型が返り値の型とされる.類似物として C 系言語の$\mathtt{void}$型がある.
    \end{frame}
\begin{frame}{Unit 型}
$\mathtt{Unit}$型は\footnote{Haskell でいうところの 0-tuple},以下の構文,型付け規則および派生形式(i.e.糖衣構文),すなわち STLC への埋め込み方で定義される.
\begin{dblock}{Def.$\mathtt{Unit}$型の定義}
	\begin{columns}
		\begin{column}{0.30\textwidth}
			新しい構文形式
			\begin{align*}
			\mathtt{\text{(項)\ }t}\Coloneq&\ldots\\
                                        &\mathtt{Unit}
			\end{align*}
        \begin{align*}
                \mathtt{\text{(項)\ }v}\Coloneq&\ldots\\
                &\mathtt{unit}
            \end{align*}
            \begin{align*}
                \mathtt{\text{(型)\ }T}\Coloneq&\ldots\\
                &\mathtt{Unit}
            \end{align*}
		\end{column}\begin{column}{0.66\textwidth}
			新しい型付け規則\begin{prooftree}
				\AxiomC{$\Gamma\vdash \mathtt{unit:Unit}$}
			\end{prooftree}
            新しい派生形式
            \begin{align*}
                \mathtt{t_{1};t_{2}}&\stackrel{\text{def}}{=}(\lambda \mathtt{x:Unit. t_{2}})\mathtt{t_{1}}\\
                    &\text{ただし}x\notin FV(\mathtt{t_{2}})
            \end{align*}
		\end{column}
	\end{columns}\end{dblock}
\end{frame}
\begin{frame}{\S11.3.派生形式: 逐次実行とワイルドカード}
副作用のある言語における文の逐次実行を形式化する.逐次実行形式は項$\mathtt{t_{1}, t_{2}}$に対し,$\mathtt{t_{1}}$を正規形まで評価し,結果を捨てた後に$\mathtt{t_{2}}$を評価する.

$(;)$の意味を直接表わす方法と,$(;)$を内部言語のある項の略記とする 2 通りの形式化が考えられる.\begin{enumerate}\item $\mathtt{t_{1};t_{2}}$を新たな構文要素とする方法.評価規則
\begin{prooftree}
\AxiomC{$\mathtt{t_{1}\rightarrow t'_{1}}$}
\RightLabel{\tiny\rm{(E-SEQ)}}
\UnaryInfC{$\mathtt{t_{1};t_{2}\rightarrow t'_{1};t_{2}}$}
\end{prooftree}
\begin{prooftree}
        \AxiomC{}
        \noLine
        \UnaryInfC{$\mathtt{v_{1};t_{2}\rightarrow t_{2}}$\ \tiny\rm{(E-SEQNEXT)}}
\end{prooftree}
および型付け規則
\begin{prooftree}
\AxiomC{$\Gamma\vdash\mathtt{t_{1}:Unit}$}
\AxiomC{$\Gamma\vdash\mathtt{t_{2}:T_{2}}$}
\RightLabel{\tiny\rm{(T-SEQ)}}
\BinaryInfC{$\Gamma\vdash\mathtt{t_{1};t_{2}:T_{2}}$}
\end{prooftree}を付け加えることで$(;)$の振る舞いを特徴付ける.
\item 内部言語の項の略記とする方法.$\mathtt{t_{1};t_{2}\stackrel{def}{=}(\lambda x:Unit. t_{2})t_{1}}$,ただし$x\mathtt{x}\notin FV(\mathtt{t_{2}})$とする.
\end{enumerate}
\end{frame}
\begin{frame}{派生形式}
はじめの形式化で定めた評価および型付け規則は,$\mathtt{Unit}$のみを型として持つ STLC の評価関係および型付け規則より従う.このことを確認しよう.
\begin{alertblock}{Thm.11.3.1[逐次実行は派生形式である]}
    $\mathtt{Unit}$型,逐次実行およびそれらの評価・型付け規則を持つ STLC を$\lambda^{E}$と書く.\footnote{外部言語の略.}また,$\mathtt{Unit}$型のみを持つ STLC を$\lambda^{I}$と書き,$e\in\lambda^{I}\to\lambda^{E}$を,$\lambda^{I}$の各項を対応する$\lambda^{E}$の項に写す詳細化関数\footnote{elaboration function}とする.つまり,$e$は$\mathtt{t_{1};t_{2}}$の各出現を,$\mathtt{(\lambda x:Unit. t_{2})t_{1}}(\mathtt{x}\notin FV(\mathtt{t_{2}}))$に置き換える.すると,$\lambda^{E}$の各項$\mathtt{t}$に対して,\begin{enumerate}
    \item $\mathtt{t\rightarrow_{E} t'}\Rightarrow e(\mathtt{t})\rightarrow_{I} e(\mathtt{t'})$.逆に$e(\mathtt{t})\rightarrow_{I}\mathtt{u}\Rightarrow\exists \mathtt{t'}.\mathtt{t'}$は$\lambda^{E}$の項$\land \mathtt{u} = e(\mathtt{t'})\land \mathtt{t\rightarrow_{E}t'}$.
    \item $\Gamma\vdash^{E}\mathtt{t:T}\Leftrightarrow \Gamma\vdash^{I}e\mathtt{(t):T}$
    \end{enumerate}
\end{alertblock}
\end{frame}
\begin{frame}{Thm.11.3.1.の証明(一部)}
    \begin{proof}
       $\mathtt{t}$の構造帰納法で示す.(2)の主張は白板で示すことにする.
        \begin{enumerate}
        \item (1)の証明.\begin{enumerate}
        \item $(\Rightarrow)$ 新たな構文要素,つまり$\mathtt{t = t_{1};t_{2}}$または$\mathtt{t = v_{1};t_{2}}$の場合を考え,$\mathtt{t\rightarrow_{E} t'}$なる$\mathtt{t'}$の存在を仮定.\begin{itemize}
        \item $\mathtt{t = v_{1};t_{2}}$.仮定より$\mathtt{t' = t_{2}}$.いま$\mathtt{x}\notin FV(\mathtt{t_{2}})$を任意に取ると,定義より$e(\mathtt{t}) = \mathtt{(\lambda x: Unit. }e\mathtt{(t_{2}))\ v_{1}}$.$\text{\rm(E-APPABS)}$から$\mathtt{t}\rightarrow_{I}\left[\mathtt{x\mapsto v_{1}}\right]e\mathtt{(t_{2})} = \mathtt{t'}$がいえ,$\mathtt{x\notin}FV(e(\mathtt{t_{2}}))$からこれは$\mathtt{v_{1}}$に等しい.
        \item $\mathtt{t = t_{1};t_{2}}$.仮定より$\mathtt{t' = t'_{1};t_{2}}\land \mathtt{t_{1}\rightarrow_{E}t'_{1}}$なる$\mathtt{t', t'_{1}}$が存在.IH より$e(\mathtt{t_{1}}) \rightarrow_{I} e(\mathtt{t'_{1}})$.$\text{\rm{(E-APP1)}}$から$e(\mathtt{t}) = e(\mathtt{t_{1}})e(\mathtt{t_{2}})\rightarrow_{I} e(\mathtt{t'_{1}})e(\mathtt{t_{2}} = e(\mathtt{t'})$.
        \end{itemize}
        \item $(\Leftarrow)$.$e(\mathtt{t})\rightarrow_{I}\mathtt{u}$を仮定する.(1)と同様に場合を分ける.\begin{itemize}
        \item $\mathtt{t = v_{1};t_{2}}$のとき.$e(\mathtt{t}) = \mathtt{(\lambda x:Unit. }e\mathtt{(t_{2}))v_{1}}$($\mathtt{x}$は fresh な変数),$\mathtt{u} = e(\mathtt{t_{2}})$.このとき.$\mathtt{t' = t_{2}}$とすれば従う.
        \item $\mathtt{t = t_{1};t_{2}}$のとき.$\mathtt{t} = (\mathtt{\lambda x:Unit.}e (\mathtt{t_{2}}))\ e(\mathtt{t_{1}})$であり,仮定より$e(\mathtt{t_{1}})\rightarrow \mathtt{u'_{1}}$なる$\mathtt{u'_{1}}$を取ると$\mathtt{u} = (\mathtt{\lambda x:Unit.}e(\mathtt{t_{1}}))\ \mathtt{u'_{1}}$.帰納法の仮定より,$\mathtt{u'_{1}} = e(\mathtt{t'_{1}})\land \mathtt{t_{1}}\rightarrow_{E} \mathtt{t'_{1}}$なる$\lambda^{E}$の項$\mathtt{t'_{1}}$が取れる.ゆえに$\mathtt{u} = (\mathtt{\lambda x: Unit.}e(\mathtt{t_{2}}))e(\mathtt{t'_{1}}) = e(\mathtt{t'_{1};t_{2}})$となり従う.
        \end{itemize}
        \end{enumerate}
        \end{enumerate}
    \end{proof}
\end{frame}
\begin{frame}{\S11.4.型指定}
項の型を明示的に指定する機能を導入しよう.\begin{alertblock}{Def. 型指定}
型$\mathtt{T}$を指定した項$\mathtt{t}$を,$\mathtt{t\ as\ T}$と書く.評価規則を次のように定める:
\infax[E-Ascribe]{\mathtt{v_{1}\ as\ T\longrightarrow v_{1}}}
\infrule[E-Ascribe1]{\mathtt{t_{1}\longrightarrow t'_{1}}}{\mathtt{t_{1}\ as\ T\longrightarrow t'_{1}\ as\ T}}
型付け規則を次のように定める:
\infrule[T-Ascribe1]{\Gamma\vdash\mathtt{t_{1}:T}}{\Gamma\vdash\mathtt{t_{1}\ as\ T:T}}
\end{alertblock}
\end{frame}
\begin{frame}{演習 11.4.1.}
\begin{alertblock}{演習 11.4.1.[推奨]}
\begin{enumerate}
\item 型指定を派生形式として形式化せよ.「公式」の型付け規則および評価規則が,その形式化に何らかの適切な意味で対応することを証明せよ.
\item Def.で定めた評価規則の代わりに,以下の先行的な規則が与えられたとする:\infax[E-AscribeEager]{\mathtt{t_{1}\ as\ T\longrightarrow t_{1}}}.この場合にも,型指定を派生形式として扱うことはできるか.
\end{enumerate}
\end{alertblock}
\end{frame}
\begin{frame}{Ex11.4.1.の解答}
\begin{enumerate}
  \item $\mathtt{t\ as\ T}$の派生形式は$\mathtt{(\lambda x:T.x)\ t}$とすればよい.また,拡張された構文の項を内部言語の項に写す関数$e$を次のように定める:
\begin{equation*}e(t) = \begin{cases}
\mathtt{(\lambda x:T. x)\ t_{1}}& \text{$\mathtt{t}$が$\mathtt{t_{1}:T}$の形のとき} \\
\mathtt{t} & \text{otherwise}
\end{cases}\end{equation*}.$e$は明らかに全単射である.「公式」の型付けおよび評価規則が派生形式と対応することを示すためには,次の補題を示せばよい.
\begin{alertblock}{Cor.}\begin{enumerate}
\item $\mathtt{t\longrightarrow t'}$のとき$\mathtt{e(t)\longrightarrow e(t')}$.また,$\mathtt{e(t)\longrightarrow u}$ならば型指定された項$\mathtt{t'}$が存在して$\mathtt{u = e(t')\land t\longrightarrow t'}$
\item (型付け)$\Gamma\vdash\mathtt{t:T}$,かつそのときに限り$\Gamma\vdash \mathtt{e(t):T}$
\end{enumerate}
\end{alertblock}
\end{enumerate}
\end{frame}
\begin{frame}{Ex11.4.1:補題の証明}
\begin{enumerate}
\item 前半部分は,$\mathtt{t}$を$\mathtt{t = t_{1}:T}$の形に限定し,$\mathtt{t_{1}}$が値か否かで分ければ示せる.後半部分についても$\mathtt{t}$が型指定された項$\mathtt{t_{1}\ as\ T}$の場合のみに限定し,$\mathtt{e(t) = (\lambda x:T. x)t_{1}}$,$\mathtt{u}$は$\mathtt{t_{1}}$が値のとき$\mathtt{t_{1}}$,そうでないとき$\mathtt{(\lambda x:T. x)t'_{1}}$(ただし$\mathtt{t_{1}\longrightarrow t'_{1}}$)に評価されることに注意すれば,評価された項に$e$の逆関数を適用して存在が示せる.
\item $\Gamma\vdash\mathtt{t\ as\ T}$の型導出は次のようになるので,$\Gamma\vdash\mathtt{t:T}$が成立:\begin{prooftree}\AxiomC{$\vdots$}\UnaryInfC{$\Gamma\vdash\mathtt{t:T}$}\UnaryInfC{$\Gamma\vdash\mathtt{t\ as\ T:T}$}\end{prooftree}
$\Gamma\vdash\mathtt{e(t\ as\ T):T}$に至る導出木は,$\Gamma\vdash\mathtt{t:T}$に至る部分木から下の導出木を作ることで示せる:\begin{prooftree}
\AxiomC{$\Gamma,\mathtt{x:T}\vdash\mathtt{x:T}$}
\UnaryInfC{$\Gamma\vdash\mathtt{\lambda x:T.x:T\to T}$}
\AxiomC{$\vdots$}
\UnaryInfC{$\Gamma\vdash\mathtt{t:T}$}
\BinaryInfC{$\Gamma\vdash \mathtt{e(t\ as\ T) = (\lambda x:T.x)\ t:T}$}
\end{prooftree}

逆向きの主張は,上の導出木の部分木で$\Gamma\vdash\mathtt{t:T}$が成立していることから,$\tiny\rn{T-Ascribe}$を用いて示せる.
\end{enumerate}
\end{frame}
\begin{frame}{Ex11.4.1(2)}
派生形式として表わすことができる.型指定された項の評価を進めることなく消去するので,次のように定義すればよい:\begin{equation}
\mathtt{t\ as\ T\stackrel{def}{=}(\lambda x:Unit\to T.x\ Unit)(\lambda y:Unit.t_{1})}
\end{equation}
\end{frame}
\begin{frame}{\S11.5.let 束縛}
部分式に対して名付けるための機能である$\mathtt{let}$束縛子を導入する.
\begin{alertblock}{Def.let 束縛}
式$\mathtt{t_{1}}$を評価した値を変数$\mathtt{x}$に束縛した状態で項$\mathtt{t_{2}}$を評価する.これを,式$\mathtt{let\ x=t_{1}\ in\ t_{2}}$で表わす.評価規則を次のように定める:
\infax[E-LetV]{\mathtt{let\ x=v_{1}\ in\ t_{2}\longrightarrow \left[x\mapsto v_{1}\right]t_{2}}}
\infrule[E-Let]{\mathtt{t_{1}\longrightarrow t'_{1}}}{\mathtt{let\ x=t_{1}\ in\ t_{2}\longrightarrow let\ x=t'_{1}\ in\ t_{2}}}

つまり,この定義における$\mathtt{let}$束縛子は値呼びである.型付け規則を次のように定める:
\infrule[T-Let]{\Gamma\vdash\mathtt{t_{1}:T_{1}}\andalso \Gamma,\mathtt{x:T_{1}}\vdash \mathtt{t_{2}:T_{2}}}{\Gamma\vdash\mathtt{let\ x=t_{1}\ in\ t_{2}}}
つまり,束縛項の型を計算し,その型を持つ束縛で水増しされた文脈で本体の型を計算することで,$\mathtt{let}$式全体の型を計算している.
\end{alertblock}
\end{frame}
\begin{frame}{$\mathtt{let}$束縛に関する補足}
\begin{enumerate}
\item $\mathtt{let}$式は派生形式$\mathtt{let\ x=t_{1}\ in\ t_{2} \stackrel{def}{=}(\lambda x:T_{1}.t_{2})t_{1}}$として書けるが,束縛項の型は型検査器からしか分からない.
\begin{itemize}
\item 型指定などのように項を脱糖するのではない(構文的には束縛項の型が分からない!)
\item 派生形式は型付け導出\begin{prooftree}
\AxiomC{$\vdots$}
\UnaryInfC{$\Gamma\vdash\mathtt{t_{1}:T_{1}}$}
\AxiomC{$\vdots$}
\UnaryInfC{$\Gamma, \mathtt{x:T_{1}}\vdash\mathtt{t_{2}:T_{2}}$}
\BinaryInfC{$\Gamma\vdash\mathtt{let\ x=t_{1}\ in\ t_{2}}$}\end{prooftree}を\begin{prooftree}
\AxiomC{$\vdots$}
\UnaryInfC{$\Gamma,\mathtt{x:T_{1}\vdash t_{2}:T_{2}}$}
\UnaryInfC{$\Gamma\vdash\mathtt{\lambda x:T_{1}.t_{2}:T_{1}\to T_{2}}$}
\AxiomC{$\vdots$}
\UnaryInfC{$\Gamma\vdash\mathtt{t_{1}:T_{1}}$}
\BinaryInfC{$\Gamma\vdash\mathtt{(\lambda x:T_{1}.t_{2})t_{1}:T_{2}}$}
\end{prooftree}に写す変換とみなす必要がある.
\item また,今後$\mathtt{let}$多相を導入するさいに問題が生じる(後述).$\mathtt{let}$は型検査器によって多相的な定義を一般化する(i.e.全称量化子で束縛する)ように扱われるため.
\end{itemize}
\end{enumerate}
\end{frame}
\begin{frame}{\S11.7.直積}
型$T_{1},\ldots,T_{n}$に対して直積型$\{\mathtt{T_{i}}^{i\in1\ldots n}\}$と,直積型の項$t:\{\mathtt{T_{i}}^{1\in1\ldots n}\}$と各$j = 1,\ldots, n$に対する射影演算子$t.j:\{\mathtt{T_{i}}^{i\in1\ldots n}\}\to T_{j}$を導入する.
\begin{alertblock}{Def.$n$-tuple}
$n$項の直積の導入に伴う項,値および型の構文要素を次のように定める:\begin{enumerate}
\item (項)$\mathtt{t\Coloneq\ldots|\{t_{i}^{i\in 1\ldots n}\}|t.i}$
\item (値)$\mathtt{v\Coloneq\ldots|\{\mathtt{v_{i}^{i\in1\ldots n}}\}}$
\item (型)$\mathtt{T\Coloneq\ldots|\{\mathtt{T_{i}^{i\in1\ldots n}}\}}$
\end{enumerate}
評価規則を次のように定める:
\begin{columns}
\begin{column}{0.5\columnwidth}
\infax[E-ProjTuple]{\{\mathtt{v_{i}^{i\in1\ldots n}\}.j\longrightarrow v_{j}}}
\infrule[E-Proj]{\mathtt{t_{1}\longrightarrow t'_{1}}}{\mathtt{t_{1}.i\longrightarrow t'_{1}.i}}
\end{column}
\begin{column}{0.5\columnwidth}
\infrule[E-Tuple]{\mathtt{t_{j}\longrightarrow t'_{j}}}{\{\mathtt{v_{i}^{i\in1\ldots j-1}}, t_{j}, \mathtt{t_{k}^{k\in j+1\ldots n}}\}\\\longrightarrow \{\mathtt{v_{i}^{i\in1\ldots j- 1}},t'_{j},\mathtt{t_{k}^{k\in j + 1\ldots n}}\}}
\end{column}
\end{columns}
\end{alertblock}
\end{frame}
\begin{frame}{\S11.7.(つづき)}
\begin{alertblock}{Def.$n$-tuple(つづき)}
型付け規則は以下の 2 つ:
\infrule[T-Tuple]{\forall i. \Gamma\vdash\mathtt{t_{i}:T_{i}}}{\Gamma\vdash\{\mathtt{t_{i}^{i\in1\ldots n}\}:\{T_{i}^{i\in1\ldots n}\}}}
\infrule[T-Proj]{\Gamma\vdash\mathtt{t_{1}:\{T_{i}^{i\in1\ldots n}\}}}{\Gamma\vdash\mathtt{t_{1}.j:T_{j}}}
\end{alertblock}
\dnote{規則$\rn{E-ProjTuple}$は,次のように書き直せる:
\infrule[E-ProjTuple]{1\leq j\leq n}{\mathtt{\{v_{i}^{i\in 1\ldots j - 1}, v_{j},v_{k}^{k\in j + 1\ldots n}\}.j\longrightarrow v_{j}}}}
\end{frame}
\begin{frame}{\S11.6.二つ組}
型$\mathtt{T_{1}, T_{2}}$に対する直積型$\mathtt{T_{1}\times T_{2}}$と,直積型の項$\mathtt{t = \left\{t_{1},t_{2}\right\}:T_{1}\times T_{2}}$に対する射影演算子$\mathtt{t.1:T_{1}\times T_{2}\to T_{1}},\mathtt{t.2:T_{1}\times T_{2}\to T_{2}}$を導入する.
\begin{alertblock}{Def.2-tuple}
2 つ組の導入に伴う項,値および型の構文要素を次のように定める:\begin{enumerate}
\item $\mathtt{t\Coloneq\ldots|\left\{t, t\right\}|t.1|t.2}$
\item $\mathtt{v\Coloneq\ldots|\left\{v,v\right\}}$
\item $\mathtt{T\Coloneq\ldots|T\times T}$
\end{enumerate}
評価規則を次のように定める.組は第一成分から評価を進める.
\infax[E-PairBeta1]{\mathtt{\left\{v_{1},v_{2}\right\}.1\longrightarrow v_{1}}}
\infax[E-PairBeta2]{\mathtt{\left\{v_{1}, v_{2}\right\}.2\longrightarrow v_{2}}}
\end{alertblock}\end{frame}
\begin{frame}{\S11.6}
\begin{alertblock}{Def.2-tuple(続き)}
\begin{columns}
\begin{column}{0.5\columnwidth}
\infrule[E-Proj1]{\mathtt{t_{1}\longrightarrow t'_{1}}}{\mathtt{t_{1}.1\longrightarrow t'_{1}.1}}
\infrule[E-Proj2]{\mathtt{t_{1}\longrightarrow t'_{1}}}{\mathtt{t_{1}.2\longrightarrow t'_{1}.2}}
\infrule[E-Pair1]{\mathtt{t_{1}\longrightarrow t'_{1}}}{\mathtt{\left\{t_{1},t_{2}\right\}\longrightarrow \left\{t'_{1},t_{2}\right\}}}
\infrule[E-Pair2]{\mathtt{t_{2}\longrightarrow t'_{2}}}{\mathtt{\left\{v_{1},t_{2}\right\}\longrightarrow \left\{v_{1},t'_{2}\right\}}}
\end{column}
\begin{column}{0.5\columnwidth}
型付け規則
\infrule[T-Pair]{\Gamma\vdash\mathtt{t_{1}:T_{1}}\andalso \Gamma\vdash\mathtt{t_{2}:T_{2}}}{\Gamma\vdash\mathtt{\left\{t_{1},t_{2}\right\}:T_{1}\times T_{2}}}
\infrule[T-Proj1]{\Gamma\vdash\mathtt{t_{1}:T_{11}\times T_{12}}}{\Gamma\vdash\mathtt{t_{1}.1:T_{11}}}
\infrule[T-Proj2]{\Gamma\vdash\mathtt{t_{1}:T_{11}\times T_{12}}}{\Gamma\vdash\mathtt{t_{1}.2:T_{12}}}
\end{column}
\end{columns}
\end{alertblock}
\end{frame}
\begin{frame}{\S11.6.コメント}
評価規則により,二つ組は第一成分が値に評価されたのちに第二成分が評価される.たとえば,$\mathtt{t = \left\{pred\ 4,if\ true\ then\ false\ then\ false\right\}.1}$の簡約列は次のように定まる:
\begin{align*}
\mathtt{t}&\longrightarrow \mathtt{\left\{3, if\ true\ then\ true\ else\ false\right\}}.1\\
  &\longrightarrow \mathtt{\left\{3, true\right\}.1}\longrightarrow 3
\end{align*}

値の構文を追加することで,関数の引数として渡された 2 つ組は本体の評価前に完全に評価される.
\end{frame}
\begin{frame}{\S11.8.組}
$n$-タプルに対し,各フィールド$\mathtt{t_{i}}$に集合$\mathscr{L}$から選んだラベル$\mathtt{l_{j}}$を注記することで,ラベル付きレコードに一般化する.次のように定義する:
\begin{alertblock}{Def.レコード}
新しい構文形式:\begin{enumerate}
\item $\mathtt{t\Coloneq\ldots\{\l_{i}=t_{i}^{i\in1\ldots n}\}|t.l}$
\item $\mathtt{v\Coloneq\ldots|\{\l_{i}=v_{i}^{i\in1\ldots n}\}}$
\item $\mathtt{T\Coloneq\ldots|\{t_{i}:T_{i}^{i\in1\ldots n}\}}$
\end{enumerate}
\end{alertblock}
\end{frame}
\begin{frame}{\S11.8.(つづき)}
\begin{alertblock}{Def.レコード(つづき)}
\begin{columns}
\begin{column}{0.5\columnwidth}
評価規則は以下の通り.
\infax[E-ProjRcd]{\mathtt{\{l_{i}=v_{i}^{i\in1\ldots n}\}.l_{j}\longrightarrow v_{j}}}
\infrule[E-Proj]{\mathtt{t_{1}\longrightarrow t'_{1}}}{\mathtt{t_{1}.l\longrightarrow t'_{1}.l}}
\infrule[E-Rcd]{\mathtt{t_{j}\longrightarrow t'_{j}}}{\{\mathtt{l_{i}=v_{i}^{i\in1\ldots j- 1}, l_{j} = t_{j},l_{k}=t_{k}^{k\in j+1\ldots n}}\}\\\longrightarrow \{\mathtt{l_{i}=v_{i}^{i\in1\ldots j - 1}, l_{j} = t'_{j},l_{k}=t_{k}^{k\in j + 1\ldots n}}\}}

\end{column}
\begin{column}{0.5\columnwidth}
型付け規則は次の通り.
\infrule[T-Rcd]{\forall i, \Gamma\vdash\mathtt{t_{i}:T_{i}}}{\Gamma\vdash\{\mathtt{l_{i}=t_{i}^{i\in1\ldots n}}\}\{\mathtt{T_{i}^{i\in1\ldots n}}\}}
\infrule[T-Proj]{\Gamma\vdash t_{1}:\{\mathtt{t_{i}:T_{i}^{i\in1\ldots n}}\}}{\Gamma\vdash\mathtt{t_{1}.l_{j}:T_{j}}}
\end{column}
\end{columns}
\end{alertblock}
\dnote{$\rn{E-ProjRcd}$は次のように書き直せる:\infrule[E-ProjRcd]{1\leq j \leq n \andalso \mathtt{l_{j}}\in\mathscr{L}}{\{\mathtt{l_{i}= v_{i}^{i\in1\ldots n}\}.l_{j}\longrightarrow v_{j}}}}
\end{frame}
\begin{frame}{\S11.8.レコードパターン}
射影操作を用いずにフィールドを取り出す方法としてパターンマッチがある.STLC 上では次のように定義できる(Ex.11.8.2)
\begin{alertblock}{Def.レコードパターン}
\begin{columns}
\begin{column}{0.5\columnwidth}
新しい構文形式.$\mathtt{p}$を変数パターンとする:\begin{align*}
\mathtt{p}&\Coloneq \mathtt{x|\{l_{i}=p_{i}^{{i\in1\ldots n}}\}}\\
\mathtt{t}&\Coloneq\ldots\mathtt{let\ p=t\ in\ t}\\
\end{align*}
パターンマッチの規則.
\infax[M-Var]{\mathtt{match\ (x, v)=[x\mapsto v]}}
\infrule[M-Rcd]{\forall i.\mathtt{match\ (p_{i}, v_{i})}=\sigma_{i}}{\mathtt{match(\{l_{i}=p_{i}^{i\in1\ldots n}\},\{l_{i}=v_{i}^{i=1\ldots n}\})}\\= \sigma_{i}\circ\cdots\circ\sigma_{n}}
\end{column}
\begin{column}{0.5\columnwidth}
パターン型付け規則:
\infax[P-Var]{\vdash\mathtt{x:T\Rightarrow x:T}}
\infrule[P-Rcd]{\forall i.\mathtt{\vdash p_{i}:T_{i}}\Rightarrow \Delta_{i}}{\vdash\mathtt{\{l_{i}= p_{i}^{i\in1\ldots n}\}:\{k_{j}:T_{j}^{j\in1\ldots m}}\\\Rightarrow \Delta_{i},\ldots \Delta_{n}}
型付け規則:
\infrule[T-Let]{\Gamma\vdash\mathtt{t_{1}:T_{1}}\andalso \vdash\mathtt{p:T_{1}}\Rightarrow \Delta\\\andalso \Gamma,\Delta\vdash\vdash\mathtt{t_{2}:T_{2}}}{\Gamma\vdash\mathtt{let\ p=t_{1}\ in\ t_{2}:T_{2}}}
\end{column}
\end{columns}
\end{alertblock}
\end{frame}
\begin{frame}{\S11.8.レコードパターン}
評価規則:
\infax[E-LetV]{\mathtt{let\ p=v_{1}\ in\ t_{2}}\longrightarrow \mathtt{match(p, v_{1})\ t_{2}}}
\infrule[E-Let]{\mathtt{t_{1}\longrightarrow t'_{1}}}{\mathtt{let\ p=t_{1}\ in\ t_{2}}\longrightarrow\mathtt{let\ p=t'_{1}\ in\ t_{2}}}
\end{frame}
\begin{frame}{\S11.9.和}
型$\mathtt{T_{1},T_{2}}$に対する(二項の)和型$\mathtt{T_{1}+T_{2}}$と,項$\mathtt{t_{1}:T_{1}}$($\mathtt{t_{2}:T_{2}}$)から$\mathtt{T_{1}+T_{2}}$の項を構成するタグ$\mathtt{inl}(\mathtt{inr})$,そして和型上のパターンマッチのための構文を導入する.\footnote{$\mathtt{inl,inr}$は関数でない.}
\begin{alertblock}{Def.和型}
新しい構文形式\begin{enumerate}
\item (項)$\mathtt{t\Coloneq\ldots|inl\ t|inr\ t|case\ t\ of\ inl\ x\Rightarrow t|inr\ x\Rightarrow t}$
\item (値)$\mathtt{v\Coloneq\ldots|inl\ v|inr\ v}$
\item (型)$\mathtt{T\Coloneq\ldots|T+T}$
\end{enumerate}
評価規則を次のように定める:
\infax[E-CaseInl]{\mathtt{case(inl\ v_{0})\ of\ inl\ x_{1}\Rightarrow t_{1}}\\\mathtt{|inr\ x_{2}\Rightarrow t_{2}}\\\longrightarrow \mathtt{[x_{1}\mapsto v_{0}]t_{1}}}
\infax[E-CaseInr]{\mathtt{case(inr\ v_{0})\ of\ inl\ x_{1}\Rightarrow t_{1}}\\\mathtt{|inr\ x_{2}\Rightarrow t_{2}}\\\longrightarrow \mathtt{[x_{2}\mapsto v_{0}]t_{2}}}
\end{alertblock}
\end{frame}
\begin{frame}{\S11.9}
\begin{alertblock}{Def.和型}
\begin{columns}
\begin{column}{0.5\columnwidth}
\infrule[E-Case]{\mathtt{t_{0}\longrightarrow t'_{0}}}{\mathtt{case\ t_{0}\ of\ inl\ x_{1}\Rightarrow t_{1}|inr\ x_{2}\Rightarrow t_{2}}\\\longrightarrow \mathtt{case\ t'_{0}\ of\ inl\ x_{1}\Rightarrow t_{1}|inr\ x_{2}\Rightarrow t_{2}}}
\infrule[E-Inl]{\mathtt{t_{1}\longrightarrow t'_{1}}}{\mathtt{inl\ t_{1}\longrightarrow inl\ t'_{1}}}
\infrule[E-Inr]{\mathtt{t_{1}\longrightarrow t'_{1}}}{\mathtt{inr\ t_{1}\longrightarrow inr\ t'_{1}}}
\end{column}
\begin{column}{0.5\columnwidth}
\infrule[T-Inl]{\Gamma\vdash\mathtt{t_{1}:T_{1}}}{\Gamma\vdash\mathtt{inl\ t_{1}:T_{1}+T_{2}}}
\infrule[T-inr]{\Gamma\vdash\mathtt{t_{1}:T_{2}}}{\Gamma\vdash\mathtt{inr\ t_{1}:T_{1}+T_{2}}}
\end{column}
\end{columns}
\infrule[T-Case]{\Gamma\vdash\mathtt{t_{0}:T_{1}+T_{2}}\andalso \Gamma,\mathtt{x_{1}:T_{1}}\vdash\mathtt{t_{1}:T}\andalso \Gamma,\mathtt{x_{2}:T_{2}}\vdash\mathtt{t_{2}:T}}{\Gamma\vdash\mathtt{case\ t_{0}\ of\ inl\ x_{1}\Rightarrow t_{1}|inr\ x_{2}\Rightarrow t_{2}:T}}
\end{alertblock}
\end{frame}
\begin{frame}{\S11.9.和と型の一意性}
型付け規則$\rn{T-Inl}(\rn{T-Inr})$によれば,型導出の仮定に含まれない型には任意性がある.したがって,型の一意性(Thm.9.3.3)は成立しない.一意性を取り戻す方法としては,3 つ考えられる:\begin{enumerate}
\item 型検査アルゴリズムに変更を加え,不定となっている型を推論させる(\S22 で扱う.)
\item 型の言語を改良し,可能な全ての$\mathtt{T_{2}}$を統一的に表現できるようにする.部分型を用いる.
\item 型注釈を付ける.今回はこちらを用いる.
\end{enumerate}
評価および型付け規則は,全頁に示した規則に加え,型注釈つきの和型に対する規則を加えればよい.
\end{frame}
\begin{frame}{\S11.9.}
\begin{alertblock}{Def.和(一意型付け)}
\begin{columns}
\begin{column}{0.5\columnwidth}
新しい構文形式\begin{itemize}
\item 項:$\mathtt{t\Coloneq\ldots|inl\ t\ as\ T|inr\ t\ as\ T}$
\item 値:$\mathtt{v\Coloneq\ldots|inl\ v\ as\ T|inr\ v\ as\ T}$
\end{itemize}
新しい評価規則
\infax[E-CaseInl]{\mathtt{case\ (inl\ v_{0}\ as\ T_{0})\ of\ inl\ x_{1}\Rightarrow t_{1}}\\\mathtt{|inr\ x_{2}\Rightarrow t_{2}}\\\longrightarrow\mathtt{[x_{1}\mapsto v_{0}]t_{1}}}
\infax[E-CaseInr]{\mathtt{case\ (inr\ v_{0}\ as\ T_{0})\ of\ inl\ x_{1}\Rightarrow t_{1}}\\\mathtt{|inr\ x_{2}\Rightarrow t_{2}}\\\longrightarrow\mathtt{[x_{2}\mapsto v_{0}]t_{2}}}
\end{column}
\begin{column}{0.5\columnwidth}
\infrule[E-Inl]{\mathtt{t_{1}\longrightarrow t'_{1}}}{\mathtt{inl\ t_{1}\ as\ T_{2}\longrightarrow inl\ t'_{1}\ as\ T_{2}}}
\infrule[E-Inr]{\mathtt{t_{1}\longrightarrow t'_{1}}}{\mathtt{inr\ t_{1}\ as\ T_{2}\longrightarrow inr\ t'_{2}\ as\ T_{2}}}
型付け規則
\infrule[T-Inl]{\Gamma\vdash\mathtt{t_{1}:T_{1}}}{\Gamma\vdash\mathtt{inl\ t_{1}\ as\ T_{1}+T_{2}:T_{1}+T_{2}}}
\infrule[T-Inr]{\Gamma\vdash\mathtt{t_{1}:T_{2}}}{\Gamma\vdash\mathtt{inr\ t_{1}\ as\ T_{1}+T_{2}:T_{1}+T_{2}}}
\end{column}
\end{columns}
\end{alertblock}
\end{frame}
\begin{frame}{\S11.10.バリアント}
二項の和型を一般化するとラベル付きバリアントを得る.構文,評価および型付け規則は型指定付きの和型を拡張し,次のように定める.\begin{alertblock}{Def.バリアント}
新しい構文形式:\begin{align*}
\mathtt{t}\Coloneq&\mathtt{\ldots|\langle l=t\rangle\ as\ T|case\ t\ of\ \langle l_{i}=x_{i}\rangle\Rightarrow t_{i}^{i\in1\ldots n}}\\
\mathtt{v}&\Coloneq\mathtt{\ldots|\langle l=v\rangle\ as\ T}\\
\mathtt{T}&\Coloneq\mathtt{\ldots|\langle l_{i}:T_{i}^{i\in1\ldots n}\rangle}\\
\end{align*}
評価規則を次のように定める:
\infax[E-CaseVariant]{\mathtt{case\ (\langle l_{j}=v_{j}\rangle as\ T)\ of\ \langle l_{i}=x_{i}\rangle\Rightarrow t_{i}^{i\in1\ldots n}}\\\longrightarrow \mathtt{[x_{j}\mapsto v_{j}]t_{j}}}
\infrule[E-Case]{\mathtt{t_{0}\longrightarrow t'_{0}}}{\mathtt{case\ t_{0}\ of\ \langle l_{i} =x_{i}\rangle\Rightarrow t_{i}^{i\in1\ldots n}}\\\longrightarrow\mathtt{case\ t'_{0}\ of\ \langle l_{i}=x_{i}\rangle\Rightarrow t_{i}^{i\in1\ldots n}}}
\infrule[E-Variant]{\mathtt{t_{i}\longrightarrow t'_{i}}}{\mathtt{<l_{i}=t_{i}>\ as\ T\longrightarrow <l_{i}=t'_{i}>\ as\ T}}
\end{alertblock}
\end{frame}
\begin{frame}{\S11.10}
\begin{alertblock}{}
型付け規則を次のように定める:
\infrule[T-Variant]{\Gamma\vdash\mathtt{t_{j}:T_{j}}}{\Gamma\vdash\mathtt{\langle l_{j}=t_{j}\rangle\ as\ \langle l_{i}:T_{i}^{i\in1\ldots n}\rangle:\langle l_{i}:T_{i}^{i\in1\ldots n}\rangle}}
\infrule[T-Case]{\Gamma\vdash\mathtt{t_{0}:\langle l_{i}:T_{i}^{i\in1\ldots n}\rangle}\\\andalso \forall i. \Gamma,\mathtt{x_{i}:T_{i}}\vdash\mathtt{t_{i}:T}}{\Gamma\vdash\mathtt{case\ t_{0}\ of\ \langle l_{i}=x_{i}\rangle\Rightarrow t_{i}^{i\in1\ldots n}:T}}
\end{alertblock}
\end{frame}
\begin{frame}{\S11.10.}
ヴァリアント型の特殊なケースとして,オプション型,列挙型および単一フィールドのヴァリアント型を考えよう.
\begin{enumerate}
\item オプション値.ML 系言語の$\mathtt{option}$や$\mathtt{Maybe}$型に対応し,各データ型に特別な値$\mathtt{none}$を追加する.\begin{itemize}\item $\mathtt{nat\ option}$型と同型な$\mathtt{OptionalNat}$は$\mathtt{OptionalNat = \langle none:Unit, some:Nat\rangle}$と書ける.型コンストラクタ$\mathtt{option}$は,$\mathtt{option(T) = \langle none:Unit, some:T\rangle}$と定義すればよい.
\item C 系言語の$\mathtt{null}$値もオプションと見做せるが,$\mathtt{null}$値は参照型をもつので,$\mathtt{Ref(Option(T))}$なる型を持つ.
\end{itemize}
\item 列挙型は,各フィールドの型が$\mathtt{Unit}$型である,いわば退化したヴァリアント型である.\begin{itemize}
\item OCaml などではフィールドの型は省略できる.
\end{itemize}
\item 単一フィールドのヴァリアントは通常の型($\mathtt{Float}$など)をラップする.\footnote{Haskell における$\mathtt{newtype}$宣言など.}値に対して何らかの制約を課したい場合に使われる.\begin{itemize}
\item 値を単一フィールドのヴァリアントとすることで,たとえばユーロからドルに換算するプログラムが単位ユーロの値のみを取るようにできる($\mathtt{Float}$を$\mathtt{EuroAmount =\langle euros:Float\rangle}$のように包むことで)
\end{itemize}
\end{enumerate}
\end{frame}
\begin{frame}{\S11.10.ヴァリアント型の拡張}
この他にも,ヴァリアント型にいくつかの拡張を施したものが使われている.
\begin{enumerate}
\item 再帰データ型.列挙型のフィールドに定義されている型が出現を持つ.具体例には自然数のリスト型$\mathtt{type\ NatList= Nil|Cons\ of\ Nat\ *\ NatList}$がある.
\item パラメータ化されたデータ型.たとえば,(多相的)リスト型$\mathtt{type\ 'a\ List = nil|cons\ of\ 'a\ *\ 'a\ List}$など.$\mathtt{List}$は各型$\mathtt{T}$を($\mathtt{'a}$を$\mathtt{T}$に具体化した)型$\mathtt{T\ List}$に写すので,型演算子と呼ばれる.
\end{enumerate}
\end{frame}
\begin{frame}{\S11.12.リスト}
新たな型構築子として,型$\mathtt{T}$の要素からなる有限長のリストを表わる型$\mathtt{List\ T}$を追加する.構文,意味論および型付けを次のように定める.\begin{alertblock}{Def.リスト}
構文形式:
\begin{equation*}\mathtt{T\Coloneq\ldots|nil[T]|cons[T]\ t\ t|isnil[T]\ t|head[T]\ t|tail[T]\ t}
\end{equation*}
\begin{equation*}\mathtt{v\Coloneq\ldots|nil[T]|cons[T]\ v\ v}
\end{equation*}
\begin{equation*}\mathtt{T\Coloneq\ldots|List\ T}
\end{equation*}
\end{alertblock}
\end{frame}
\begin{frame}{\S11.12.}
\begin{alertblock}{}
\begin{columns}
\begin{column}{0.5\columnwidth}
評価規則は次の通り.
\infrule[E-Cons1]{\mathtt{t_{1}\longrightarrow t'_{1}}}{\mathtt{cons[T]\ t_{1}\ t_{2}\longrightarrow cons[T]\ t'_{1}\ t_{2}}}
\infrule[E-Cons2]{\mathtt{t_{2}\longrightarrow t'_{2}}}{\mathtt{cons[T]\ v_{1}\ t_{2}\longrightarrow cons[T]\ v_{1}\ t'_{2}}}
\infax[E-IsNilNil]{\mathtt{isnil[S](nil[T])\longrightarrow false}}
\infax[E-IsNilCons]{\mathtt{isnil[S](cons[T]\ v_{1}\ v_{2})\longrightarrow false}}
\end{column}
\begin{column}{0.5\columnwidth}
\infrule[E-IsNil]{\mathtt{t_{1}\longrightarrow t'_{1}}}{\mathtt{isnil[T]\ t_{1}\longrightarrow isnil[T]\ t'_{1}}}
\infax[E-HeadCons]{\mathtt{head[S](cons[T]\ v_{1}\ v_{2})\longrightarrow v_{1}}}
\infrule[E-Head]{\mathtt{t_{1}\longrightarrow t'_{1}}}{\mathtt{head[T]\ t_{1}\longrightarrow head[T]\ t'_{1}}}
\infax[E-TailCons]{\mathtt{tail[S]\ (cons[T]\ v_{1}\ v_{2})\longrightarrow v_{2}}}
\infrule[E-Tail]{\mathtt{t_{1}\longrightarrow t'_{1}}}{\mathtt{tail[T]\ t_{1}\longrightarrow tail[T]\ t'_{1}}}
\end{column}
\end{columns}\end{alertblock}
\end{frame}
\begin{frame}{\S11.12}
\begin{alertblock}{Def.リスト型の型付け規則.}
\infax[T-Nil]{\Gamma\vdash\mathtt{nil[T_{1}]:List\ T_{1}}}
\infrule[T-Cons]{\Gamma\vdash\mathtt{t_{1}:T_{1}}\andalso \Gamma\vdash\mathtt{t_{2}:List\ T_{1}}}{\Gamma\vdash\mathtt{cons[T_{1}]\ t_{1} \ t_{2}:List\ T_{1}}}
\infrule[T-Head]{\Gamma\vdash\mathtt{t_{1}:List\ T_{11}}}{\Gamma\vdash\mathtt{head[T_{11}]\ t_{1}:T_{11}}}
\infrule[T-Tail]{\Gamma\vdash\mathtt{t_{1}:List\ T_{11}}}{\Gamma\vdash\mathtt{tail[T_{11}]\ t_{1}:List\ T_{11}}}
\end{alertblock}
\end{frame}
\begin{frame}{\S11.10.$\mathtt{Dynamic}$型}
型がコンパイル時にわからないデータを扱う場合がある.この場合は値$\mathtt{v}$と型タグ$\mathtt{T}$(ただし$\mathtt{v:T}$)の組を値とする$\mathtt{Dynamic}$型を用いる.
\end{frame}
\begin{frame}{\S11.11.一般的再帰}
STLC を拡張し,再帰関数を定義できるようにしよう.STLC では$\mathtt{fix}$コンビネータ自体を定義できない\footnote{$\mathtt{f:T_{1}\to T_{1}}$のもとで不動点コンビネータ$\mathtt{fix=\lambda f.(\lambda x. f\ (\lambda y. x\ x\ y))(\lambda x.f\ (\lambda y. x\ x\ y))}$が型付けできたとすると,基底型を持つはずの$\mathtt{x}$が同時に関数型を持つことになるため.}が,以下のプリミティブ$\mathtt{fix}$および略記法$\mathtt{letrec}$
を導入することで再帰関数を定義できる.\footnote{この時点で強正規化性は失われる.後述}
\begin{alertblock}{Def.一般的再帰}
新しい構文形式:
\begin{equation*}
\mathtt{t\Coloneq\ldots|fix\ t}
\end{equation*}\begin{columns}
\begin{column}{0.5\columnwidth}
評価規則を次のように定める:
\infax[E-FixBeta]{\mathtt{fix\ (\lambda x:T_{1}.t_{2})}\\\longrightarrow\mathtt{[x\mapsto (fix\ (\lambda x:T_{1}.t_{2})]t_{2}}}
\infrule[E-Fix]{\mathtt{t_{1}\longrightarrow t'_{1}}}{\mathtt{fix\ t_{1}\longrightarrow fix\ t'_{1}}}
\end{column}
\begin{column}{0.5\columnwidth}
型付け規則を次のように定める:
\infrule[T-Fix]{\Gamma\vdash\mathtt{t_{1}:T_{1}\to T_{1}}}{\Gamma\vdash\mathtt{fix\ t_{1}:T_{1}}}

また,派生形式$\mathtt{letrec}$は次の通り:
\infax{\mathtt{letrec\ x:T_{1}=t_{1}\ in\ t_{2}}\stackrel{def}{=}\\\mathtt{let\ x\ =\ fix\ (\lambda x:T_{1}.t_{1})\ in\ t_{2}}}
\end{column}
\end{columns}
\end{alertblock}
\end{frame}
\begin{frame}{\S11.11.具体例}
偶数ならば$\mathtt{true}$を,そうでないなら$\mathtt{false}$を返す関数$\mathtt{iseven:Nat\to Bool}$は以下の$\mathtt{ff:(Nat\to Bool)\to Nat\to Bool}$を用いて$\mathtt{iszero = fix\ ff}$と定義される:\begin{align*}
\mathtt{ff}= \mathtt{\lambda ie:Nat\to Bool.}&\mathtt{\lambda x:Nat. if\ iszero\ x\ then\ true}\\
  &\mathtt{else\ if\ iszero(pred\ x)\ then\ false}\\
&\mathtt{else\ ie\ (pred(pred\ x))}
\end{align*}
ここで,$\mathtt{ff}$は$\mathtt{iseven}$関数の生成子と見做せる.\begin{itemize}
\item $\mathtt{ff}$が$n$以下の自然数に対して$\mathtt{iseven}$を近似する関数$\mathtt{ie}$に適用されると,$\mathtt{ff}$は$n +2$まで$\mathtt{iseven}$を近似する関数を返す.
\item $\mathtt{fix}$を$\mathtt{ff}$に適用することで,生成子の不動点が返される\footnote{関数上に適切な半順序を定めれば,この不動点が順序関係の上限であることがわかる.また,証明はしないが,この不動点は最小不動点である.}
                                                                \end{itemize}
\end{frame}
\begin{frame}{\S11.11.コメント}
\begin{itemize}
\item STLC に数と再帰を追加した言語を PCF という.
\item 型付け規則$\rn{T-Fix}$における生成子の型は任意なので,生成子をレコード上の関数とすれば相互再帰を実現できる.
\item また,規則$\rn{T-Fix}$から,任意の型$\mathtt{T}$に対して$\mathtt{T\to T}$型の関数の不動点が取れる.\begin{itemize}
\item ここから,任意の型$\mathtt{T}$は inhabited であることが従う.各型$\mathtt{T}$に対して,関数$\mathtt{diverge}_{T}:\mathtt{Unit\to T}$を$\mathtt{diverge}_{T} = \mathtt{\lambda \_:Unit.fix\ (\lambda x:T. x)}$が定まる.これを$\mathtt{unit}$に適用すると発散する.
\end{itemize}
\end{itemize}
\end{frame}
\section{\S12.正規化}
\begin{frame}{\S12.正規化}
この章では,任意の正しく型付けされた STLC の項が正規化可能であることを示す.\footnote{ここでは値呼び戦略のみ考えるが,より強い主張である強正規化性も示せる.}
\begin{itemize}
\item 正規化可能性は一般に成り立つ性質ではない.STLC に一般的再帰や再帰型を追加して拡張した場合,評価(型チェック)が停止しない項を構成できるから.
\item ただし,高階$\lambda$計算の体系である System $F_{\omega}$の型検査の停止性を議論する際,型言語が STLC と等価であることを用いる.
\end{itemize}
はじめに基本型として$A$のみを持つ STLC について議論し,その後ブール値と直積で拡張した体系に対して議論する.
\end{frame}
\begin{frame}{\S12.1.単純型の正規化}
\begin{alertblock}{Thm.正規化定理}
任意の正しく型付けされた STLC の項の評価は正規化可能である.
\end{alertblock}
まずは今までの道具で証明できないか考える.たとえば,well-typed な項のサイズに関する帰納法で証明してみる.\begin{itemize}
\item well-typed な$\mathtt{t}$が,正規形でない項$\mathtt{t_{1}.t_{2}}$により$\mathtt{t_{1}\ t_{2}}$と書ける場合.
\begin{itemize}
\item (IH)より$\mathtt{t_{1},t_{2}}$の nf として$\mathtt{v_{1},v_{2}}$が存在する.型付けの逆転補題により$\mathtt{t_{1}:T_{1}\to T_{2}, t_{2}:T_{1}}$なる型$\mathtt{T_{1},T_{2}}$が取れる.
\item したがって$\mathtt{v_{1}}$は関数型を持つので,$\mathtt{v_{1}}$は$\mathtt{v_{1}=\lambda x:T_{1}.t_{12}}$の形.
\item よって$\mathtt{t_{1}\ t_{2}}$は$\mathtt{v_{1}\ v_{2} = (\lambda x:T_{1}.t_{12})v_{2}}$まで簡約されるが,これは正規形でない.
\item IH がこの項が正規化可能であることを示すほど強くない.たとえば$\mathtt{t_{12} = \lambda y:T_{2}\to T_{2}\to T_{2}. \lambda z:T_{2}\to T_{2}. y\ x\ (z\ y)}$の形の場合,項のサイズは明らかに$\mathtt{t}$より大きい.
\end{itemize}
\end{itemize}
\end{frame}
\begin{frame}{\S12.1.飽和集合}
induction の仮定を強めよう.そのために,各型$\mathtt{T}$に対し,型$\mathtt{T}$の閉じた項からなる集合$R_{\mathtt{T}}$を定める.この集合を飽和集合と呼ぶことにする.また,閉項$\mathtt{t:T}$が$\mathtt{t}\in R_{\mathtt{T}}$なるとき,単に$R_{\mathtt{T}}(\mathtt{t})$と書くことにする.
\begin{alertblock}{Def.飽和集合}
型$\mathtt{T}$の閉項からなる集合$R_{\mathtt{T}}$を次のように定める:\begin{itemize}
\item $R_{\mathtt{A}}(\mathtt{t})\Leftrightarrow\mathtt{t}$が停止
\item $R_{\mathtt{T_{1}\to T_{2}}}(t)\Leftrightarrow\mathtt{t}$が停止し,任意の$\mathtt{s}\in R_{\mathtt{T_{1}}}$に対し$R_{\mathtt{T_{2}}}(\mathtt{t\ s})$.
\end{itemize}
\end{alertblock}
飽和集合を用いることで,閉項の強正規化可能性を示すのに十分強い仮定が得られる.\begin{itemize}
\item 閉項は関数型の部分項を持ちうるので,その情報も必要である.
\item 部分項の正規化可能性では不十分で,関数適用後の正規化可能性も示す必要がある.正規形の関数に正規形の引数を適用した代入したのち,さらに評価が進む可能性があるため.
\end{itemize}
\end{frame}
\begin{frame}{\S12.1.証明の概略}
飽和集合を用いた強正規化性の証明は次のように進む.型に関する induction で進める:\begin{enumerate}
\item 型$\mathtt{A}$の任意の項は正規化可能.
                                                                                  \item 任意の型$\mathtt{T_{1}, T_{2}}$と任意の項$\mathtt{t:T_{1}\to T_{2}}$に対し,$\mathtt{t}$は強正規化可能性を保つ.つまり,$\mathtt{t}$は正規化可能,かつ任意の正規化可能な$\mathtt{s:T_{1}}$に対し,$\mathtt{t\ s}$は正規化可能.
\end{enumerate}
それぞれ分けて証明する(Lem.12.1.3, Lem.12.1.4, Lem.12.1.5).
\end{frame}
\begin{frame}{Lem.12.1.3, Lem.12.1.4 の証明}
\begin{alertblock}{Lem.12.1.3}
任意の$R_{\mathtt{T}}(\mathtt{t})$なる$\mathtt{t}$に対して,$\mathtt{t}$は停止する.
\end{alertblock}これは定義から明らか.次に,項$\mathtt{t}$が$R_{\mathtt{T}}$に属することは評価に対して不変であることを示す.
\begin{alertblock}{Lem.12.1.4}
$\mathtt{t:T}\land \mathtt{t\to t'}$ならば$R_{\mathtt{T}}(\mathtt{t})\Leftrightarrow R_{\mathtt{T}}(\mathtt{t'})$
\end{alertblock}
\begin{proof}
$\mathtt{T}$の構造帰納法で示す.\begin{itemize}\item $\mathtt{T}$が基本型の場合は明らか.
\item $\mathtt{T}$が関数型,つまりある$\mathtt{T_{1}, T_{2}}$により$\mathtt{T = T_{1}\to T_{2}}$と書ける場合.\begin{enumerate}
\item $(\Rightarrow)$.定義より,$\mathtt{t}$は正規化可能かつ$\forall \mathtt{s}\in R_{\mathtt{T_{1}}}.R_{\mathtt{T_{2}}}(\mathtt{t\ s})$.よって$\mathtt{t'}$も正規化可能.また,各$\mathtt{s}\in R_{\mathtt{T_{1}}}$に対し$\mathtt{t\ s}\longrightarrow\mathtt{t'\ s}$より,(IH)から$\mathtt{t'\ s}\in R_{\mathtt{T_{2}}}$.ゆえに$R_{{\mathtt{T_{1}\to T_{2}}}}(\mathtt{t'})$.
\item $(\Leftarrow)$.$\mathtt{t'}\in R_{{\mathtt{T_{1}\to T_{2}}}}$より,$\mathtt{t'}$は正規化可能かつ$\forall \mathtt{s}\in R_{\mathtt{T_{1}}}.R_{\mathtt{T_{2}}}(\mathtt{t'\ s})$.仮定より$\mathtt{t\longrightarrow t'}$なので$\mathtt{t}$は正規化可能.また,各$\mathtt{s'}\in R_{\mathtt{T_{1}}}$に対し,$\mathtt{t\ s}\stackrel{\rn{E-App1}}{\longrightarrow}\mathtt{t'\ s}$なので$R_{\mathtt{T_{2}}}(\mathtt{t\ s})$.ゆえに$R_{\mathtt{T_{1}\to T_{2}}}(\mathtt{t})$
\end{enumerate}
\end{itemize}
\end{proof}
\end{frame}
\begin{frame}{Lem.12.1.6 の証明}
Lem.12.1.3.より,正規化定理を論理関係を用いて証明するには,型$\mathtt{t}$の全ての項が飽和集合$R_{\mathtt{T}}$に属することをいえばよい:
\begin{alertblock}{Thm.12.1.6[正規化]}
$\vdash\mathtt{t:T}$ならば,$\mathtt{t}\in R_{\mathtt{T}}$
\end{alertblock}証明は型付け導出に関する induction に依る.
閉じた項$\mathtt{t}$が型付け可能とする.最後に使われた型付け規則で分ける.
\begin{itemize}
\item $\rn{T-Var}$のとき.不可能.
\item $\rn{T-App}$のとき.逆転補題および帰納法の仮定から,$\mathtt{t_{1}:T_{1}\to T, t_{2}:T_{1}}$かつ$\mathtt{R_{\mathtt{T_{1}}}(t_{1}), R_{\mathtt{T_{1}\to T}}(t_{2})}$なる$\mathtt{t_{1}, t_{2}}$により$\mathtt{t = t_{1}\ t_{2}}$と書ける.$\mathtt{R_{\mathtt{T_{1}\to T}}}$の定義より,$\forall\mathtt{s}\in R_{\mathtt{T_{1}}}.R_{\mathtt{T}}(\mathtt{t_{1}\ s})$を用いれば,$\mathtt{t}\in R_{\mathtt{T}}$は即座に従う.
\item $\rn{T-Abs}$のとき.項$\mathtt{t_{2}}$,変数$\mathtt{x}\in FV(\mathtt{t_{2}})$
と$\mathtt{x:T_{1}}\vdash \mathtt{t_{2}:T_{2}}$なる$\mathtt{T_{2}}$が存在して$\mathtt{t = \lambda x:T_{1}.t_{2}}$.このとき$\mathtt{T= T_{1}\to T_{2}}$である.$\mathtt{t_{2}}$は閉項ではないので,次頁の Lem.12.1.5.を用いて$R_{\mathtt{T_{2}}}(\mathtt{t_{2}})$なることを示す.
\end{itemize}
\end{frame}
\begin{frame}{Lem.12.1.5.}
\begin{alertblock}{Lem.12.1.5}
$\mathtt{x_{1}:T_{1},\ldots x_{n}:T_{n}\vdash\mathtt{t:T}}$かつ$\mathtt{v_{1}:T_{1},\ldots v_{n}:T_{n}}$が閉項で,かつ任意の$\mathtt{i}$に対して$R_{\mathtt{T_{i}}}(\mathtt{v_{i}})$ならば,$R_{\mathtt{T}}(\mathtt{[x_{1}\mapsto v_{1}]\ldots[x_{n}\mapsto v_{n}]t})$
\end{alertblock}
証明は型付け導出の induction による.各$i$について$R_{\mathtt{T_{i}}}(\mathtt{v_{i}})$を満たす閉じた項$\mathtt{v_{1}:T_{1},\ldots v_{n}:T_{n}}$を任意に取る.以下場合分け.\begin{enumerate}
\item $\rn{T-Var}$のときは$\exists i.\mathtt{t = x_{i}:T_{i}}$であり明らか.
\item $\rn{T-App}$のとき.\begin{itemize}
\item $\mathtt{t = t_{1}\ t_{2}}$.
\item 帰納法の仮定より\begin{itemize}
\item $(IHt_{1}): \mathtt{x_{1}:T_{1},\ldots x_{n}:T_{n}\vdash t_{1}:T_{11}\to T_{12}\Rightarrow R_{T_{11}\to T_{12}}(\mathtt{[x_{1}\mapsto v_{1}]\ldots [x_{n}\mapsto v_{n}]t_{1}})}$
\item $(IHt_{2}): \mathtt{x_{1}:T_{1},\ldots x_{n}:T_{n}\vdash t_{2}:T_{11}\Rightarrow R_{\mathtt{T_{11}}}(\mathtt{[x_{1}\mapsto v_{1}]\ldots[x_{n}\mapsto v_{n}]t_{2}})}$
\end{itemize}
\item $\mathtt{x_{1}:T_{1},\ldots x_{n}:T_{n}\vdash t_{1}: T_{11}\to T_{12}, t_{2}:T_{11}}$が成り立つので,$R_{\mathtt{T_{11}\to T_{12}}}(\mathtt{[x_{1}\mapsto v_{1}]\ldots[x_{n}\mapsto v_{n}]t_{1}}), R_{\mathtt{T_{11}}}(\mathtt{[x_{1}\mapsto v_{1}]\ldots[x_{n}\mapsto v_{n}]t_{2}})$
\item $R_{\mathtt{T_{11}\to T_{12}}}(\_)$の定義より,題意が従う.
\end{itemize}
\end{enumerate}
\end{frame}
\begin{frame}{Lem.12.1.5}
\begin{enumerate}\setcounter{enumi}{2}
        \item $\rn{T-Abs}$の場合.\begin{itemize}
\item 次が成り立つ:\begin{itemize}
\item $\mathtt{t = \lambda x:S_{1}.s_{2}:S_{1}\to S_{2}}$
\item $\mathtt{x_{1}:T_{1},\ldots x_{n}:T_{n}, x:S_{1}\vdash s_{2}:S_{2}}$
\item $\mathtt{T = S_{1}\to S_{2}}$
\end{itemize}
\item 帰納法の仮定$(IH): \forall\mathtt{v}\in R_{\mathtt{S_{1}}}. \mathtt{x_{1}:T_{1}\ldots x_{n}:T_{n}, x:S_{1}\vdash s_{2}:S_{2}}\Rightarrow R_{\mathtt{S_{2}}}(\mathtt{[x_{1}\mapsto v_{1}]\ldots[x_{n}\mapsto v_{n}][x\mapsto v]s_{2}})$
\item $\mathtt{t}$は値,ひいては$\mathtt{[x_{1}\mapsto v_{1}]\ldots [x_{n}\mapsto v_{n}]t}$は値だから停止性が従う.
\item $\forall \mathtt{s}\in R_{\mathtt{S_{1}}}.R_{S_{2}}(\mathtt{([x_{1}\mapsto v_{1}]\ldots[x_{n}\mapsto v_{n}]t)\ s})$を示す.\begin{itemize}
\item $\mathtt{s}\in R_{\mathtt{S_{1}}}$を任意に取る.
\item Lem.12.1.3.より$\mathtt{s}$は正規化可能 i.e.$\mathtt{s\longrightarrow^{*}v}$なる正規形$\mathtt{v}$が取れる.
\item Lem.12.1.4 の系より$\mathtt{s\longrightarrow^{*}v}, R_{\mathtt{S_{1}}}(\mathtt{s})\Rightarrow R_{\mathtt{S_{1}}}(\mathtt{v})$.よって
$R_{\mathtt{S_{1}}}(\mathtt{v})$.
\item (IH)より,$R_{\mathtt{S_{2}}}(\mathtt{[x_{1}\mapsto v_{1}]\ldots[x_{n}\mapsto v_{n}][x\mapsto v]s_{2}})$
\item いま,$\mathtt{(\lambda x:S_{1}.[x_{1}\mapsto v_{1}]\ldots[x_{n}\mapsto v_{n}]s_{2})\ s\longrightarrow^{*}[x_{1}\mapsto v_{1}]\ldots[x_{n}\mapsto v_{n}][x\mapsto v] s}$
\item 再び Lem.12.1.4 の系および$\mathtt{(\lambda x:S_{1}.[x_{1}\mapsto v_{1}]\ldots[x_{n}\mapsto v_{n}]s_{2})v\longrightarrow [x_{1}\mapsto v_{1}]\ldots[x_{n}\mapsto v_{n}][x\mapsto v]s_{2}}$により,$R_{\mathtt{S_{2}}}(([x_{1}\mapsto v_{1}]\ldots[x_{n}\mapsto v_{n}]t) s)$が従う.$\mathtt{s}$は任意なので示される.
\end{itemize}
\end{itemize}
\end{enumerate}
\end{frame}
\begin{frame}{Thm.12.1.6.の証明のつづき}
\begin{itemize}
\item $\rn{T-Abs}$のとき.項$\mathtt{t_{2}}$,変数$\mathtt{x}$と$\mathtt{x:T_{1}\vdash t_{2}:T_{2}}$なる項$\mathtt{t_{2}}$が存在して$\mathtt{t = \lambda x:T_{1}.t_{2}}$.任意に$\mathtt{s}\in R_{\mathtt{T_{1}}}$を取ると,Lem.12.1.5.より$R_{\mathtt{T_{2}}}(\mathtt{[x\mapsto s]t_{2}})$.$\rn{E-AppAbs}$と Lem.12.1.4 の系より$R_{\mathtt{T_{2}}}(\mathtt{t\ s})$が従うのでこの場合も成立する.
\end{itemize}
以上より,$\vdash\mathtt{t:T}\Rightarrow R_{\mathtt{T}}(\mathtt{t})$.ゆえに Lem.12.1.3.から, $\vdash\mathtt{t:T}$ならば$\mathtt{t}$が停止することが従う.
\end{frame}
\begin{frame}{演習 12.1.7.}
\begin{alertblock}{演習 12.1.7.[推奨]}
本章の証明を拡張し,STLC がブール値と直積で拡張されても正規化できることを示せ.
\end{alertblock}
ただし,ブール値と直積の評価および型付け規則は以下の通り:\begin{columns}\begin{column}{0.5\columnwidth}
ブール式:
\infax[E-IfTrue]{\mathtt{if\ true\ then\ t_{2}\ else\ t_{3}\longrightarrow t_{2}}}
\infax[E-IfFalse]{\mathtt{if\ false\ then\ t_{2}\ else\ t_{3}\longrightarrow t_{3}}}
\infrule[E-If]{\mathtt{t_{1}\longrightarrow t'_{1}}}{\mathtt{if\ t_{1}\ then\ t_{2}\ else\ t_{3}}\\\longrightarrow\mathtt{if\ t'_{1}\ then\ t_{2}\ else\ t_{3}}}
\infax[T-True]{\mathtt{true:Bool}}
\infax[T-False]{\mathtt{false:Bool}}
\end{column}\begin{column}{0.5\columnwidth}
\infrule[T-If]{\mathtt{t_{1}:Bool} \andalso \mathtt{t_{2}:T}\andalso \mathtt{t_{3}:T}}{\mathtt{if\ t_{1}\ then\ t_{2}\ else\ t_{3}:Bool}}
\infax[E-PairBeta1]{\mathtt{\{v_{1},v_{2}\}.1\longrightarrow v_{1}}}
\infax[E-PairBeta2]{\mathtt{\{v_{1},v_{2}\}.2\longrightarrow v_{2}}}
\infrule[E-Proj1]{\mathtt{t_{1}\longrightarrow t'_{1}}}{\mathtt{t_{1}.1\longrightarrow t'_{1}.1}}
\infrule[E-Proj2]{\mathtt{t_{1}\longrightarrow t'_{1}}}{\mathtt{t_{1}.2\longrightarrow t'_{1}.2}}
\end{column}
\end{columns}
\end{frame}
\begin{frame}{Ex.12.1.7}\begin{columns}
\begin{column}{0.5\columnwidth}\infrule[Pair1]{\mathtt{t_{1}\longrightarrow t'_{1}}}{\mathtt{\{t_{1}, t_{2}\}\longrightarrow\{t'_{1}, t_{2}\}}}
\infrule[E-Pair2]{\mathtt{t_{2}\longrightarrow t'_{2}}}{\mathtt{\{v_{1}, t_{2}\}\longrightarrow\{v_{1}, t'_{2}\}}}
\infrule[T-Pair]{\Gamma\vdash\mathtt{t_{1}:T_{1}}\andalso \Gamma\vdash\mathtt{t_{2}:T_{2}}}{\Gamma\vdash\mathtt{\{t_{1},t_{2}\}:T_{1}\times T_{2}}}
\end{column}
\begin{column}{0.5\columnwidth}
\infrule[T-Proj1]{\Gamma\vdash\mathtt{t_{1}:T_{11}\times T_{12}}}{\Gamma\vdash\mathtt{t_{1}.1:T_{11}}}
\infrule[T-Proj2]{\Gamma\vdash\mathtt{t_{1}:T_{11}\times T_{12}}}{\Gamma\vdash\mathtt{t_{1}.2:T_{12}}}
\end{column}
\end{columns}
\begin{proof}
Def.12.1.2 の定義および Lem12.1.4 と Lem12.1.5.の証明の修正点を示す.\begin{itemize}
\item Def.12.1.2:ブール型と直積型の節を追加する. ブール型の場合は基本型同様.$R_{\mathtt{T_{1}\times T_{2}}}(\mathtt{t})\Leftrightarrow \mathtt{t}$が正規化可能$\land R_{\mathtt{T_{1}}}(\mathtt{t}.1)$$\land R_{\mathtt{T_{2}}}(\mathtt{t}.2)$
\end{itemize}
\end{proof}
\end{frame}
\begin{frame}{Ex.12.1.7}
\begin{proof}[Proof(cont'd)]
\begin{itemize}
\item Lem12.1.4:ブール型の場合は直積型と同様.直積型の節を追加する.\begin{itemize}
\item ある$\mathtt{T_{1},T_{2}}$により$\mathtt{T= T_{1}\times T_{2}}$となる場合を考える.$\mathtt{t\longrightarrow t'}$を仮定する.\begin{itemize}
\item $R_{\mathtt{T_{1}\times T_{2}}}(\mathtt{t})$のとき.$\mathtt{t'}$が正規化可能であることは,$\mathtt{t}$の正規形に至る簡約列が$\mathtt{t\longrightarrow t'\longrightarrow\cdots}$の形をしていることから.$R_{\mathtt{T_{1}}}(\mathtt{t.1})$($R_\mathtt{T_{2}}(\mathtt{t.2})$)は,$\rn{E-Proj1}(\rn{E-Proj2})$および帰納法の仮定より.
\item 逆向きも全く同様.
\end{itemize}
\end{itemize}
\item Lem12.1.5.項$\mathtt{t}$に対して最後に使われた型付け規則について確かめる.追加された規則について考える.$\forall i, R_{\mathtt{T_{i}}}(\mathtt{v_{i}})$なる正規形$\mathtt{v_{1}:T_{i},\ldots,v_{i}:T_{i}}$を任意に取り,各$\mathtt{x_{i}}$に$\mathtt{v_{i}}$に代入する操作を$\sigma$,文脈を$\Gamma$で表わす.
\begin{itemize}
\item $\rn{T-True},\rn{T-False}$のとき.定数に代入はできないので明らか.
\item $\rn{T-If}$のとき.$\mathtt{t}$は$\mathtt{t = if\ t_{1}\ then\ t_{2}\ else\ t_{3}:T}$の形である.示すべきは$\mathtt{t}$が正規形に評価できること.
\begin{itemize}
\item 逆転補題より$\mathtt{x_{1},\ldots, x_{n}\vdash t_{1}:Bool, t_{2}:T, t_{3}:T}$
\item (IH1):$\Gamma\vdash \mathtt{t_{1}:Bool}\Rightarrow R_{\mathtt{Bool}}(\sigma(\mathtt{t_{1}}))$
\item (IH2):$\Gamma\vdash\mathtt{t_{2}:T}\Rightarrow R_{\mathtt{T}}(\sigma(\mathtt{t_{2}}))$
\item (IH3):$\Gamma\vdash\mathtt{t_{3}:T}\Rightarrow R_{\mathtt{T}}(\sigma(\mathtt{t_{3}}))$
\item (IH1)より$\mathtt{\sigma(t_{1})}$は評価が停止していずれかのブール値になり,ブール値によって$\mathtt{\sigma(t_{2})}(\mathtt{\sigma(t_{3})})$が評価されるが,帰納法の仮定より停止性が従う.
\end{itemize}
\item 直積に関する型付け規則も同様.
\end{itemize}
\end{itemize}
\end{proof}
\end{frame}
\section{\S13.参照}
\begin{frame}{\S13.概要}
これまで扱ってきた言語機能,たとえば関数抽象,基本型やレコードなどの構造化された型は全て純粋なものだった.しかし,プログラム言語の機能には純粋でなく計算的作用を伴うもの,たとえば破壊的代入や入出力,例外や継続による非局所的な移動などが多く含まれている.

本章では,計算的作用の一つである可変参照\footnote{Java などの言語では,セルは値$\mathtt{null}$を持ちうる.したがって,$\mathtt{Ref(Option\ T)}$型を持つとみなせる.}をこれまでの計算体系に加える方法を考える.
\end{frame}
\begin{frame}{\S13.1}
\begin{alertblock}{参照への基本的操作}
参照に関して,以下 3 つの操作を考える:\begin{enumerate}
\item 確保.$\mathtt{ref}$演算子に初期値を与え,参照を確保する.参照の型は,初期値の型を$\mathtt{T}$とすると$\mathtt{Ref\ T}$型である.
\item 値の取り出し.セルの現在の値を演算子$\mathtt{!:Ref\ T\to T}$により読み出す.
\item 破壊的代入.セルに保存された値を変えるために,演算子$\mathtt{(:=):Ref\ T\to Unit}$を用いる.
\end{enumerate}
破壊的代入操作は$\mathtt{Unit}$型を持つので,代入後の値を読みだす操作は逐次実行により書ける.
\end{alertblock}

\end{frame}
\begin{frame}{\S13.1.共有状態}
\begin{itemize}
\item 別名付けが存在する可能性がある場合,たとえば$\mathtt{(r:=1\; r:=!s)}$を考える.
\item これは,$\mathtt{r,s}$が同一のセルに対する別名である場合は$\mathtt{r:=!s}$と等価でない.
\end{itemize}
\end{frame}
\end{document}
