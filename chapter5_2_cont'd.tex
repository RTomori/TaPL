%%% Local Variables:
%%% mode: japanese-laTeX
%%% TeX-engine: xetex
%%% End;
\documentclass[9pt]{beamer}
 \usepackage{zxjatype}
 \usepackage{xltxtra}
 \usepackage[ipa]{zxjafont}
\usepackage{amssymb, amsmath,amsfonts}
\usepackage{mathtools}
\usepackage{bussproofs}
\usepackage{mathcomp}
\usepackage{tcolorbox}
\tcbuselibrary{raster,skins}
\usepackage{varwidth}
\usepackage{tikz}
\usetheme{CambridgeUS}

% 色定\textmd{}義
\definecolor{mstruct}{RGB}{68, 174, 234} 
% \definecolor{malert}{RGB}{223, 153, 155}
\definecolor{malert}{RGB}{255, 76, 0}
\definecolor{mex}{RGB}{57, 149, 82}
% 見出しカラー
% block title color
% alert color
% 箇条書き
\useinnertheme{circles}
% フッダー
\setbeamertemplate{footline}[frame number]
%無を出力するコマンド
\newtcolorbox{tblock}[1]{
	enhanced, skin=enhancedlast jigsaw,
	attach boxed title to top left={xshift=-4mm,yshift=-0.5mm},
	colbacktitle=mstruct, colframe=mstruct\textmd{},
	interior style={top color=mstruct!10!white, bottom color=white},
	boxed title style={empty,arc=0pt,outer arc=0pt,boxrule=0pt},
	underlay boxed title={
		\fill[mstruct] (title.north west) -- (title.north east)
		-- +(\tcboxedtitleheight-1mm,-\tcboxedtitleheight+1mm)
		-- ([xshift=4mm,yshift=0.5mm]frame.north east) -- +(0mm,-1mm)
		-- (title.south west) -- cycle;
		\fill[mstruct!45!white!50!black] ([yshift=-0.5mm]frame.north west)
		-- +(-0.4,0) -- +(0,-0.3) -- cycle;
		\fill[mstruct!45!white!50!black] ([yshift=-0.5mm]frame.north east)
		-- +(0,-0.3) -- +(0.4,0) -- cycle; 
	},
	title=#1
}
% Definition Box
\newtcolorbox{dblock}[1]{enhanced, skin=enhancedlast jigsaw,
	attach boxed title to top left={xshift=-4mm,yshift=-0.5mm},
	colbacktitle=malert, colframe=malert,
	interior style={top color=malert!10!white, bottom color=white},
	boxed title style={empty,arc=0pt,outer arc=0pt,boxrule=0pt},
	underlay boxed title={
		\fill[malert] (title.north west) -- (title.north east)
		-- +(\tcboxedtitleheight-1mm,-\tcboxedtitleheight+1mm)
		-- ([xshift=4mm,yshift=0.5mm]frame.north east) -- +(0mm,-1mm)
		-- (title.south west) -- cycle;
		\fill[malert!45!white!50!black] ([yshift=-0.5mm]frame.north west)
		-- +(-0.4,0) -- +(0,-0.3) -- cycle;
		\fill[malert!45!white!50!black] ([yshift=-0.5mm]frame.north east)
		-- +(0,-0.3) -- +(0.4,0) -- cycle; 
	},
	title=#1
}
% subbox
\newtcolorbox{subbox}[1]{
	empty,
	coltitle=mstruct, fonttitle=\bfseries,
	borderline horizontal={0.5mm}{0pt}{mstruct},
	title=#1
	titlerule style={
		mstruct,
		arrows={Hooks[arc=270]-Hooks[arc=270]}
	}
}
\everymath{\displaystyle}
\title{Types and Programming Languages\\ Chapter 5}
\author{Rei Tomori}
\begin{document}
\maketitle
\section{5.3.形式的議論}
\begin{frame}{構文}
    \begin{block}{定義[項]}
        $\mathcal{V}$を,変数名の加算集合とする.項の集合は
    \begin{align*}
    x\in\mathcal{T} &\text{\ for\ }\forall x \in\mathcal{T}\\
    t_{1}\in \mathcal{T}, x\in\mathcal{V}&\Rightarrow \lambda x.\mathtt{t}_{1}\in\mathcal{T}
    t_{1}\in\mathcal{T}, t_{2}\in\mathcal{T}&\Rightarrow t_{1}\ t_{2}\in\mathcal{T}.
    \end{align*}
    をみたす最小の集合$\mathcal{T}$である.
\end{block}
項$\mathtt{t}$のサイズは,
\begin{align*}
    size\ x &= 1\\
    size\ (\lambda x. t') &= size(\mathtt{t'}) + 1\\
    size (\mathtt{t}_{1}\ \mathtt{t}_{2}) &= size(\mathtt{t}_{1}) + size(\mathtt{t}_{2}) + 1
\end{align*}
\end{frame}

\begin{frame}
\begin{block}{項の自由変数の集合}
    項$\mathtt{t}$の自由変数の集合$FV(\mathtt{t})$は,
    \begin{align*}
    FV(x) &= \left\{x\right\}\\
    FV(\lambda x. \mathtt{t}_{1}) &= FV(\mathtt{t}_{1})\setminus \left\{x\right\}\\
    FV(\mathtt{t}_{1}\ \mathtt{t}_{2}) &= FV(\mathtt{t}_{1})\cup FV(\mathtt{t}_{2})
    \end{align*}
\end{block}
\end{frame}
\begin{frame}{置換}
    $\beta$簡約の定義で用いた変数の置換を具体的に定義する.
\begin{block}{(置換)}
    項$\mathsf{t}$の自由変数$\mathsf{x}$を項$\mathsf{s}$による置換$\left[\mathsf{x}\mapsto \mathsf{s}\right]\mathsf{t}$を,引数$\mathsf{t}$に関する帰納的定義により次のように定める.
    \begin{align*}
    \left[\mathsf{x}\mapsto\mathsf{s}\right]\mathsf{x} &= \mathsf{s}\\
    \left[\mathsf{x}\mapsto\mathsf{s}\right]\mathsf{y} &= \mathsf{y}\\
    \left[\mathsf{x}\mapsto \mathsf{s}\right]\left(\lambda \mathsf{y}.\mathsf{t}_{1}\right)&= 
        \lambda \mathsf{y}. \left[\mathsf{x}\mapsto\mathsf{s}\right]\mathsf{t}_{1}  \text{\ if\ }x \neq y\land y\in FV(\mathsf{s})\\
        \left[\mathsf{x}\mapsto \mathsf{s}\right]\left(\mathsf{t}_{1}\ \mathsf{t}_{2}\right) &= \left[\mathsf{x}\mapsto \mathsf{s}\right]\mathsf{t}_{1}\ \left[\mathsf{x}\mapsto \mathsf{s}\right]\mathsf{t}_{2}
    \end{align*}
\end{block}
$\mathcal{T}$の$\alpha$同値類で考えているので,束縛変数と置換される変数が等しい場合は考える必要がない(評価されないため).
\end{frame}
\section{操作的意味論}
\begin{frame}
$\lambda$項の(値呼びにおける)操作的意味論は次のように定義される.
\begin{block}{型無し$\lambda$計算}
    \begin{columns}\begin{column}{0.48\textwidth}
    構文\begin{align*}
\mathsf{t}\Coloneq& &\text{項:}\\
&\mathsf{x} &\text{変数}\\
&\mathsf{\lambda\mathsf{x}.\mathsf{t}} &\text{抽象}\\
&\mathsf{t}\ \mathsf{t} &\text{適用}\\
\mathsf{v}\Coloneq & &\text{値:}\\
&\lambda\mathsf{x}.\mathsf{t} &\text{抽象}
\end{align*}\end{column}
\begin{column}{.02\textwidth}
    \rule{.1mm}{0.5\textheight}
\end{column}\begin{column}{0.48\textwidth}評価
    \begin{prooftree}
        \AxiomC{$\mathsf{t}_{1}\rightarrow \mathsf{t'}_{1}$}
        \UnaryInfC{$\mathsf{t}_{1}\ \mathsf{t}_{2}\rightarrow \mathsf{t'}_{1}\ \mathsf{t}_{2}$}
    \end{prooftree}
    \begin{prooftree}
        \AxiomC{$t_{2}\rightarrow \mathsf{t'}_{2}$}
        \UnaryInfC{$\mathsf{v}_{1}\ \mathsf{t}_{2}\rightarrow \mathsf{v}_{1}\ \mathsf{t'}_{2}$}
    \end{prooftree}
    \begin{prooftree}
        \AxiomC{$\left(\lambda\mathsf{x}.\mathsf{t}_{12}\right)\mathsf{v}_{2}\rightarrow \left[\mathsf{x}\mapsto \mathsf{v}_{2}\right]\mathsf{t}_{12}$}
    \end{prooftree}\end{column}
\end{columns}
\end{block}
\end{frame}
\begin{frame}{他の評価規則の操作的意味論}
\begin{block}{完全$\beta$簡約}
\begin{columns}
    \begin{column}{0.48\textwidth}
        \begin{align*}
        \mathsf{v}\Coloneq& &\text{値:}\\
            &\lambda \mathsf{x}.\mathsf{t}&\text{抽象の値}&
        \end{align*}
    \end{column}
    \begin{column}{.02\textwidth}
        \rule{.1mm}{0.5\textheight}
    \end{column}
    \begin{column}{0.48\textwidth}
        評価
        \begin{prooftree}
            \AxiomC{$\mathsf{t}\rightarrow\mathsf{t}'$}
            \UnaryInfC{$\lambda \mathsf{x}. t \rightarrow \lambda \mathsf{x}. t'$}
        \end{prooftree}
        \begin{prooftree}
            \AxiomC{$\mathsf{t}_{1}\rightarrow\mathsf{t}_{1}'$}
            \UnaryInfC{$\mathsf{t}_{1}\ \mathsf{t}_{2} \rightarrow \mathsf{t}_{1}'\ \mathsf{t}_{2}$}
        \end{prooftree}
        \begin{prooftree}
            \AxiomC{$\mathsf{t}_{2}\rightarrow\mathsf{t}_{2}'$}
            \UnaryInfC{$\mathsf{t}_{1}\ \mathsf{t}_{2} \rightarrow \mathsf{t}_{1}\ \mathsf{t}_{2}'$}
        \end{prooftree}
        \begin{prooftree}
            \AxiomC{$\left(\lambda\mathsf{x}.\mathsf{t}_{12}\right)\mathsf{v}_{2}\rightarrow\left[\mathsf{x}\mapsto\mathsf{v}_{2}\right]\mathsf{t}_{12}$}
        \end{prooftree}

    \end{column}
\end{columns}
\end{block}
\end{frame}
\begin{frame}{他の評価規則の操作的意味論}
    \begin{block}{正規評価順序}
    \begin{columns}
        \begin{column}{0.48\textwidth}
            \begin{align*}
            \mathsf{v}\Coloneq& &\text{値:}\\
                &\lambda \mathsf{x}.\mathsf{t}&\text{抽象の値}&
            \end{align*}
        \end{column}
        \begin{column}{.02\textwidth}
            \rule{.1mm}{0.5\textheight}
        \end{column}
        \begin{column}{0.48\textwidth}
            評価
            \begin{prooftree}
                \AxiomC{$\mathsf{t}\rightarrow\mathsf{t}'$}
                \UnaryInfC{$\lambda \mathsf{x}. t \rightarrow \lambda \mathsf{x}. t'$}
            \end{prooftree}
            \begin{prooftree}
                \AxiomC{$\mathsf{t}_{2}\rightarrow\mathsf{t}_{2}'$}
                \UnaryInfC{$\mathsf{t}_{1}\ \mathsf{t}_{2} \rightarrow \mathsf{t}_{1}\ \mathsf{t}_{2}'$}
            \end{prooftree}
            \begin{prooftree}
                \AxiomC{$\left(\lambda\mathsf{x}.\mathsf{t}_{12}\right)\mathsf{v}_{2}\rightarrow\left[\mathsf{x}\mapsto\mathsf{v}_{2}\right]\mathsf{t}_{12}$}
            \end{prooftree}
        \end{column}
    \end{columns}
    \end{block}
    \end{frame}
    \begin{frame}{他の評価規則の操作的意味論}
        \begin{block}{名前呼び}
        \begin{columns}
            \begin{column}{0.48\textwidth}
                \begin{align*}
                \mathsf{v}\Coloneq& &\text{値:}\\
                    &\lambda \mathsf{x}.\mathsf{t}&\text{抽象の値}&
                \end{align*}
            \end{column}
            \begin{column}{.02\textwidth}
                \rule{.1mm}{0.5\textheight}
            \end{column}
            \begin{column}{0.48\textwidth}
                評価
                \begin{prooftree}
                    \AxiomC{$\mathsf{t}_{2}\rightarrow\mathsf{t}_{2}'$}
                    \UnaryInfC{$\mathsf{t}_{1}\ \mathsf{t}_{2} \rightarrow \mathsf{t}_{1}\ \mathsf{t}_{2}'$}
                \end{prooftree}
                \begin{prooftree}
                    \AxiomC{$\left(\lambda\mathsf{x}.\mathsf{t}_{12}\right)\mathsf{v}_{2}\rightarrow\left[\mathsf{x}\mapsto\mathsf{v}_{2}\right]\mathsf{t}_{12}$}
                \end{prooftree}
            \end{column}
        \end{columns}
        \end{block}
        \end{frame}
        \begin{frame}{演習 5.3.7}
            演習 3.5.16 の$\mathsf{wrong}$で拡張した算術式の構文と意味論は次のようであった:
            \begin{block}{エラー項で拡張された構文(再掲)}
                \begin{columns}
                    \begin{column}{0.48\textwidth}
                        拡張された構文
                        \begin{align*}
                            \mathsf{badnat} &\Coloneq &\text{数でない正規形:}&\\
                                &\mathsf{wrong}& \text{実行時エラー}&\\
                                &\mathsf{true}& \text{定数真}&\\
                                &\mathsf{false}& \text{定数偽}&\\
                            \mathsf{badbool}&\Coloneq &\text{ブール値でない正規形:}&\\
                             &\mathsf{wrong}& \text{実行時エラー}&\\
                             &\mathsf{nv}& \text{数値}
                        \end{align*}
                    \end{column}
                    \begin{column}{.02\textwidth}
                        \rule{.1mm}{0.6\textheight}
                    \end{column}
                    \begin{column}{0.48\textwidth}
                        元の構文
                        \begin{align*}
                            \mathsf{t}\Coloneq&\ldots&\text{項}&\\
                                &\mathsf{0}&\text{定数ゼロ}&\\
                                &\mathsf{succ\ t}&\text{後者値}&\\
                                &\mathsf{pred\ t}&\text{前者値}&\\
                                &\mathsf{iszero\ t}&\text{ゼロ判定}&\\
                        \end{align*}
                        評価関係
                        \AxiomC{$\mathsf{if\ badbool\ then}\ \mathsf{t}_{1}\ \mathsf{else}\ \mathsf{t}_{2}\rightarrow \mathsf{wrong}$}
                        \DisplayProof
                        \AxiomC{$\mathsf{succ\ badnat}\rightarrow\mathsf{wrong}$}
                        \DisplayProof
                        \AxiomC{$\mathsf{pred\ badnat}\rightarrow\mathsf{wrong}$}
                        \DisplayProof
                        \AxiomC{$\mathsf{iszero\ badnat}\rightarrow\mathsf{wrong}$}
                        \DisplayProof
                    \end{column}
                \end{columns}
            \end{block}
        \end{frame}
        \begin{frame}{演習 5.3.7}
            \begin{alertblock}{問題}
               この意味論を$\lambda\mathsf{NB}$に拡張せよ.
            \end{alertblock}
            
        \end{frame}
        \begin{frame}{演習 5.3.8}
            \begin{alertblock}{問題}
               ラムダ項の評価戦略を大ステップスタイルで形式化するにはどうすればよいか示せ.
            \end{alertblock}
            
        \end{frame}
        \begin{frame}{演習 5.3.8}
            \begin{columns}
                \begin{column}{0.48\textwidth}
                    値は値に評価されるので
                    \begin{prooftree}
                        \AxiomC{}
                        \LeftLabel{\small\rm{B-V}\scriptsize\rm{ALUE:}}
                        \UnaryInfC{$v\Downarrow v$}
                        \end{prooftree}
                        \begin{prooftree}
                        \AxiomC{$t_{1}\Downarrow v_{1}$}
                        \UnaryInfC{$v_{1}\ t_{2}\Downarrow v_{1}\ t_{2}$}
                        \end{prooftree}
                \end{column}
            \end{columns}
        \end{frame}
\end{document}
